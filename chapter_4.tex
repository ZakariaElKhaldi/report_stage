\chapter*{Chapitre 4 : Réalisation et mise en pied du système}
\addcontentsline{toc}{chapter}{Chapitre 4 : Réalisation et mise en pied du système}
\thispagestyle{fancy}
\setcounter{section}{0}
\newpage

\section{Outils et technologies de développement}
Pour le développement du projet, plusieurs outils et technologies modernes ont été utilisés afin d'assurer la qualité, la performance et la maintenabilité du code.

\subsection{Environnement de développement}
L'environnement de développement a été configuré avec les outils suivants :
\begin{itemize}
  \item \textbf{Visual Studio code et neovim :} Comme éditeur de code principaux avec des extensions pour React, TypeScript et ESLint
  \item \textbf{GitHub :} Pour la gestion de versions et la collaboration
  \item \textbf{Docker :} Pour la conteneurisation des services backend
  \item \textbf{Postman :} Pour tester les API
  \item \textbf{Chrome DevTools :} Pour le debugging et l'optimisation des performances frontend
\end{itemize}

\subsection{Technologies frontend}
Pour le développement front-end de cette application, les technologies suivantes ont été utilisées :
\begin{itemize}
  \item \textbf{Next.js :} Framework React pour le rendu côté serveur et la génération de sites statiques
  \item \textbf{Tailwind CSS :} Framework CSS utilitaire pour un design responsive et personnalisable
  \item \textbf{Framer Motion :} Bibliothèque d'animations pour ajouter des transitions et effets visuels
  \item \textbf{React Hook Form :} Gestion des formulaires avec validation
  \item \textbf{TypeScript :} Pour un code plus robuste avec typage statique
\end{itemize}

\subsection{Technologies backend}
Pour le développement backend, les technologies suivantes ont été utilisées :
\begin{itemize}
  \item \textbf{Node.js :} Environnement d'exécution JavaScript côté serveur
  \item \textbf{Express :} Framework web pour Node.js
  \item \textbf{PostgreSQL :} Système de gestion de base de données relationnelle
  \item \textbf{Supabase :} Plateforme Backend-as-a-Service pour l'authentification et le stockage
  \item \textbf{Redis :} Pour la mise en cache et la gestion des sessions
\end{itemize}

