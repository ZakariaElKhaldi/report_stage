\chapter{Chapitre 2 : Cahier des Charges et Spécifications}
\thispagestyle{fancy}
\newpage

Ce chapitre présente le projet LearnExpert, ses objectifs, l'analyse des besoins fonctionnels et non fonctionnels, ainsi que les technologies choisies pour sa réalisation.

\section{Présentation du Projet LearnExpert}

\subsection{Contexte et Problématique}
Le secteur de l'éducation en ligne connaît une croissance exponentielle, accélérée encore davantage par la pandémie mondiale qui a révélé les limites des plateformes d'apprentissage existantes. Dans ce contexte, plusieurs problématiques ont été identifiées :

\begin{itemize}
  \item Les plateformes actuelles proposent souvent un contenu générique et peu personnalisé
  \item Les méthodes d'apprentissage de la programmation restent trop théoriques, avec un manque d'interactivité
  \item Les apprenants font face à une fragmentation des ressources éducatives, nécessitant de naviguer entre différentes plateformes
  \item Les outils d'apprentissage ne s'adaptent pas au rythme et au niveau de chaque utilisateur
\end{itemize}

Face à ces défis, IAAI a décidé de développer LearnExpert, une plateforme innovante centrée sur l'apprentissage de la programmation et des technologies web.

\subsection{Objectifs du Projet}
Le projet LearnExpert vise à créer une plateforme d'apprentissage complète qui se distingue par :

\begin{itemize}
  \item Une approche centrée sur la pratique et l'interactivité pour l'apprentissage de la programmation
  \item Une personnalisation de l'expérience d'apprentissage grâce à l'IA
  \item Une intégration de contenus structurés et de haute qualité pour divers langages de programmation
  \item Une architecture évolutive permettant d'ajouter de nouvelles fonctionnalités et de nouveaux contenus
  \item Une expérience utilisateur fluide et engageante
\end{itemize}

L'objectif principal est de créer une plateforme qui accompagne efficacement les apprenants dans leur parcours d'acquisition de compétences techniques, qu'ils soient débutants ou développeurs expérimentés cherchant à se perfectionner.

\section{Analyse des Besoins}

\subsection{Besoins Fonctionnels}
Les fonctionnalités clés identifiées pour la plateforme LearnExpert sont :

\begin{itemize}
  \item \textbf{Gestion des utilisateurs et authentification}
    \begin{itemize}
      \item Inscription et connexion des utilisateurs
      \item Profils utilisateurs avec suivi de progression
      \item Système de rôles (apprenant, créateur de contenu, administrateur)
    \end{itemize}
  
  \item \textbf{Gestion des cours et contenus}
    \begin{itemize}
      \item Catalogue de cours organisé par catégories et niveaux
      \item Système de modules et de leçons structurés
      \item Support pour divers formats de contenu (texte, code, vidéo)
    \end{itemize}
  
  \item \textbf{Apprentissage interactif}
    \begin{itemize}
      \item Éditeur de code intégré avec exécution en temps réel
      \item Exercices pratiques et projets guidés
      \item Tests automatisés pour valider les compétences
    \end{itemize}
  
  \item \textbf{Analyse et suivi}
    \begin{itemize}
      \item Tableau de bord de progression pour les apprenants
      \item Statistiques d'utilisation pour les administrateurs
      \item Recommandations personnalisées basées sur les performances
    \end{itemize}
  
  \item \textbf{Interface d'administration}
    \begin{itemize}
      \item Gestion des utilisateurs et des permissions
      \item Création et édition de contenu éducatif
      \item Suivi des métriques de la plateforme
    \end{itemize}
\end{itemize}

\subsection{Besoins Non Fonctionnels}
Au-delà des fonctionnalités, plusieurs exigences non fonctionnelles ont été définies :

\begin{itemize}
  \item \textbf{Performance}
    \begin{itemize}
      \item Temps de chargement des pages inférieur à 2 secondes
      \item Capacité à supporter au moins 1000 utilisateurs simultanés
      \item Exécution rapide du code dans l'environnement intégré
    \end{itemize}
  
  \item \textbf{Sécurité}
    \begin{itemize}
      \item Protection des données personnelles des utilisateurs
      \item Isolation des environnements d'exécution de code
      \item Authentification robuste et gestion sécurisée des sessions
    \end{itemize}
  
  \item \textbf{Accessibilité et UX}
    \begin{itemize}
      \item Interface intuitive et responsive (mobile, tablette, desktop)
      \item Respect des standards d'accessibilité WCAG 2.1
      \item Support multilingue (initialement français et anglais)
    \end{itemize}
  
  \item \textbf{Évolutivité et maintenance}
    \begin{itemize}
      \item Architecture modulaire facilitant l'ajout de nouvelles fonctionnalités
      \item Documentation technique complète
      \item Tests automatisés couvrant les fonctionnalités critiques
    \end{itemize}
\end{itemize}

\section{Technologies et Outils}

\subsection{Stack Technologique}
Après analyse des besoins et des contraintes du projet, les technologies suivantes ont été sélectionnées :

\begin{itemize}
  \item \textbf{Frontend}
    \begin{itemize}
      \item Next.js (framework React) pour le développement d'une application web moderne
      \item TypeScript pour garantir la robustesse et la maintenabilité du code
      \item Tailwind CSS pour un design responsive et cohérent
      \item Monaco Editor pour l'implémentation de l'éditeur de code interactif
      \item Framer Motion pour les animations et transitions fluides
    \end{itemize}
  
  \item \textbf{Backend}
    \begin{itemize}
      \item Architecture microservices avec Node.js et Express
      \item NGINX comme API Gateway pour la gestion des requêtes
      \item Supabase pour l'authentification et les services de base de données
      \item MongoDB pour le stockage des contenus de cours et des données structurées
    \end{itemize}
  
  \item \textbf{Traitement des données}
    \begin{itemize}
      \item Python pour les scripts de scraping et de nettoyage de données
      \item APIs LLM (Large Language Models) pour le traitement intelligent des contenus
      \item JSON comme format principal pour le stockage et l'échange de données
    \end{itemize}
  
  \item \textbf{Déploiement et infrastructure}
    \begin{itemize}
      \item Vercel pour l'hébergement du frontend
      \item Docker pour la conteneurisation des services backend
      \item GitHub Actions pour l'intégration et le déploiement continus
    \end{itemize}
\end{itemize}

\subsection{Environnement de Développement}
L'environnement de développement a été configuré pour optimiser la productivité et la collaboration :

\begin{itemize}
  \item Visual Studio Code comme IDE principal
  \item ESLint et Prettier pour le linting et le formatage du code
  \item Git et GitHub pour le contrôle de version et la collaboration
  \item Postman pour le test des API
  \item Figma pour la conception et le prototypage de l'interface utilisateur
\end{itemize}

Cette stack technologique a été choisie pour sa modernité, sa flexibilité et sa capacité à répondre aux besoins spécifiques du projet, tout en garantissant des performances optimales et une bonne expérience utilisateur. 