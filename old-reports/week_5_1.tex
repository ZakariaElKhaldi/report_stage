\documentclass[12pt, a4paper]{article}
% --- Packages ---
\usepackage[utf8]{inputenc}
\usepackage[T1]{fontenc}
\usepackage[french]{babel}
\usepackage{graphicx} % Make sure this is here for images
\usepackage{booktabs}
\usepackage{amsmath}
\usepackage{geometry}
\usepackage{array}
\usepackage{enumitem}
\usepackage{hyperref}
\usepackage{xcolor}
\usepackage{titlesec}
\usepackage{lmodern}
\usepackage{microtype}
\usepackage{fancyhdr}
\usepackage{listings} % Added for code/JSON display
\usepackage[scaled=0.85]{beramono} % Added for a nicer monospaced font

% --- Font Configuration ---
% --- Color Definitions ---
\definecolor{primary}{RGB}{0,51,102}
\definecolor{secondary}{RGB}{102,102,153}
\definecolor{accent}{RGB}{204,0,0}
\definecolor{codegray}{rgb}{0.5,0.5,0.5}
\definecolor{codepurple}{rgb}{0.58,0,0.82}
\definecolor{codeblue}{rgb}{0,0,0.9}
\definecolor{codegreen}{rgb}{0.1,0.6,0.1} % Darker green for comments

% --- Page Geometry ---
\geometry{
  a4paper,
  left=2.5cm,
  right=2.5cm,
  top=2.5cm,
  bottom=2.5cm,
  headheight=15pt
}
% --- Header/Footer Setup ---
\pagestyle{fancy}
\fancyhf{}
\fancyhead[L]{\small Rapport de Stage - Semaine 5 - Jour 1} % Updated
\fancyhead[R]{\small Zakaria el Khaldi}
\fancyfoot[C]{\thepage}
\renewcommand{\headrulewidth}{0.4pt}
\renewcommand{\footrulewidth}{0.4pt}
% --- Title Formatting ---
\titleformat{\section}
  {\normalfont\Large\bfseries\color{primary}}
  {\thesection}{1em}{}
\titleformat{\subsection}
  {\normalfont\large\bfseries\color{secondary}}
  {\thesubsection}{1em}{}
\titleformat{\subsubsection}
  {\normalfont\normalsize\bfseries\color{accent}}
  {\thesubsubsection}{1em}{}
% --- List Formatting ---
\setlist[itemize]{leftmargin=*, nosep}
\setlist[enumerate]{leftmargin=*, nosep}
% --- Hyperlink Setup ---
\hypersetup{
  colorlinks=true,
  linkcolor=primary,
  urlcolor=secondary,
  citecolor=accent
}

% --- Listings Setup for JSON ---
\lstdefinestyle{json}{
    language=json,
    basicstyle=\ttfamily\footnotesize,
    numbers=left,
    numberstyle=\tiny\color{codegray},
    stepnumber=1,
    numbersep=5pt,
    backgroundcolor=\color{white!95!black}, % Very light gray background
    showspaces=false,
    showstringspaces=false,
    showtabs=false,
    frame=tb, % Top and bottom frame
    framextopmargin=3pt,
    framexbottommargin=3pt,
    rulecolor=\color{black!30!white},
    tabsize=2,
    captionpos=b,
    breaklines=true,
    breakatwhitespace=false,
    stringstyle=\color{codepurple},
    commentstyle=\color{codegreen},
    keywordstyle=\color{codeblue}, % For true, false, null
    morestring=[b]",
    literate=
     *{0}{{{\color{codeblue}0}}}{1}
      {1}{{{\color{codeblue}1}}}{1}
      {2}{{{\color{codeblue}2}}}{1}
      {3}{{{\color{codeblue}3}}}{1}
      {4}{{{\color{codeblue}4}}}{1}
      {5}{{{\color{codeblue}5}}}{1}
      {6}{{{\color{codeblue}6}}}{1}
      {7}{{{\color{codeblue}7}}}{1}
      {8}{{{\color{codeblue}8}}}{1}
      {9}{{{\color{codeblue}9}}}{1}
      {:}{{{\color{black}:}}}{1}
      {\{}{{{\color{black}{\{}}}}{1}
      {\}}{{{\color{black}{\}}}}}{1}
      {[}{{{\color{black}{[}}}}{1}
      {]}{{{\color{black}{]}}}}{1}
      {,}{{{\color{black}{,}}}}{1},
}


% --- Title Page Information ---
\title{\Huge\bfseries\color{primary} Rapport de Stage \\ 
      \Large Semaine 5 - Jour 1 : Infrastructure, Service IAM et Tests} % Updated title
\author{\Large Zakaria el Khaldi}
\date{\large Le 3 juin 2025} % Updated date for Day 1, Week 5 (Monday)

% --- Document Start ---
\begin{document}
% --- Cover Page ---
\begin{titlepage}
  \centering
  \vspace*{\stretch{0.5}}
  {\Huge\bfseries\color{primary} Rapport de Stage \par}
  \vspace{1cm}
  {\Large\itshape Semaine 5 - Jour 1 : Configuration Nginx/Docker, Développement du Service IAM et Préparation des Tests\par} % Updated title
  \vspace{2cm}
  
  \vspace{2cm}
  {\Large Zakaria el Khaldi\par}
  \vfill
  {\large Le 1 juin 2025\par} % Date of activity day
  \vspace*{\stretch{1}}
\end{titlepage}

% --- Table of Contents ---
\tableofcontents
\thispagestyle{empty}
\newpage

% --- Introduction ---
\section{Introduction}
\thispagestyle{fancy}
Ce rapport quotidien entame la cinquième semaine de stage. La journée a été principalement axée sur des tâches d'infrastructure et de développement backend cruciales pour la robustesse et la scalabilité de l'application LearnExpert. Les efforts se sont concentrés sur la configuration de l'environnement de déploiement avec Nginx et Docker, ainsi que sur l'initiation du développement du service de gestion des identités et des accès (IAM). Parallèlement, la préparation des tests pour les nouvelles fonctionnalités a été amorcée, et le travail sur la page d'accueil globale s'est poursuivi.

% --- Day's Accomplishments ---
\section{Activités du Jour (Lundi 1 Juin 2025)} % Updated day and date

\subsection{Configuration de Nginx et de l'Environnement Docker}
Une part importante de la journée a été consacrée à la mise en place de l'infrastructure de base pour le déploiement et la gestion des services.
\begin{itemize}
    \item \textbf{Configuration de Nginx :} Nginx a été configuré pour agir comme un reverse proxy. Cela permettra de gérer efficacement les requêtes entrantes, de distribuer la charge et de faciliter la gestion des certificats SSL à l'avenir.
    \item \textbf{Mise en place des Fichiers Docker :} Des Dockerfiles ont été créés ou adaptés pour les différents composants de l'application. L'objectif est de containeriser les services pour assurer la cohérence des environnements de développement, de test et de production, et pour simplifier le déploiement.
    \item Ces étapes sont fondamentales pour préparer l'application à une architecture microservices ou, du moins, à une séparation claire des préoccupations.
\end{itemize}

\subsection{Développement du Service d'Authentification et de Gestion des Identités (IAM)}
Le travail a commencé sur le service IAM, un composant central pour la gestion des utilisateurs de la plateforme.
\begin{itemize}
    \item \textbf{Conception des Points de Terminaison (Endpoints) :} Les efforts initiaux se sont portés sur la définition et l'implémentation des deux points de terminaison les plus critiques :
    \begin{itemize}
        \item \texttt{/register} : Pour permettre aux nouveaux utilisateurs de créer un compte.
        \item \texttt{/login} : Pour permettre aux utilisateurs existants de s'authentifier.
    \end{itemize}
    \item \textbf{Logique de Base :} La logique métier initiale pour la création de compte (validation des entrées, hachage des mots de passe) et pour la vérification des identifiants de connexion a été mise en œuvre.
    \item Ce service est essentiel pour sécuriser l'accès à la plateforme et pour personnaliser l'expérience utilisateur.
\end{itemize}

\subsection{Préparation et Rédaction des Tests}
En parallèle du développement des points de terminaison IAM, la rédaction des cas de tests a été initiée.
\begin{itemize}
    \item Des tests unitaires et d'intégration sont en cours de préparation pour valider le fonctionnement correct des endpoints \texttt{/register} et \texttt{/login}.
    \item L'objectif est de couvrir les scénarios nominaux ainsi que les cas d'erreur (par exemple, identifiants incorrects, données d'inscription invalides).
    \item Cette approche proactive des tests vise à garantir la fiabilité du service IAM dès ses premières phases de développement. Les tests seront exécutés et affinés lors de la prochaine session.
\end{itemize}


\subsection{Planification pour le Jour Suivant (Mardi 3 Juin 2025)} % Corrected date
Pour la journée de demain, les objectifs seront axés sur la page d'accueil globale :
\begin{itemize}
  \item Poursuivre activement le développement de la page d'accueil globale (landing page).
  \item Rechercher et collecter davantage d'inspiration visuelle pour affiner la conception de la page.
  \item Se concentrer spécifiquement sur l'amélioration et la finalisation de la section "hero" (partie principale visible au premier chargement).
  \item S'assurer que l'esthétique et le message de la section "hero" sont en parfaite adéquation avec l'identité de marque du projet (par exemple, "iaai" si c'est le nom de la marque à mettre en avant).
  \item Commencer l'intégration des éléments visuels et du contenu définis pour la section "hero".
\end{itemize}

\section{Conclusion}
Cette première journée de la cinquième semaine a été marquée par des avancées significatives sur des aspects fondamentaux du projet LearnExpert. La mise en place de Nginx et Docker jette les bases d'une infrastructure solide, tandis que le démarrage du service IAM est une étape cruciale pour la gestion des utilisateurs. L'anticipation des tests pour ces nouvelles fonctionnalités témoigne d'une approche méthodique visant la qualité. La poursuite du travail sur la page d'accueil assure que l'expérience utilisateur reste une priorité. Les tâches entreprises aujourd'hui sont essentielles pour la scalabilité, la sécurité et la fonctionnalité globale de la plateforme.

\end{document}