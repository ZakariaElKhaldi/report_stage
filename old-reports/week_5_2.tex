\documentclass[12pt, a4paper]{article}
% --- Packages ---
\usepackage[utf8]{inputenc}
\usepackage[T1]{fontenc}
\usepackage[french]{babel}
\usepackage{graphicx} % Make sure this is here for images
\usepackage{booktabs}
\usepackage{amsmath}
\usepackage{geometry}
\usepackage{array}
\usepackage{enumitem}
\usepackage{hyperref}
\usepackage{xcolor}
\usepackage{titlesec}
\usepackage{lmodern}
\usepackage{microtype}
\usepackage{fancyhdr}
\usepackage{listings} % Added for code/JSON display
\usepackage[scaled=0.85]{beramono} % Added for a nicer monospaced font

% --- Font Configuration ---
% --- Color Definitions ---
\definecolor{primary}{RGB}{0,51,102}
\definecolor{secondary}{RGB}{102,102,153}
\definecolor{accent}{RGB}{204,0,0}
\definecolor{codegray}{rgb}{0.5,0.5,0.5}
\definecolor{codepurple}{rgb}{0.58,0,0.82}
\definecolor{codeblue}{rgb}{0,0,0.9}
\definecolor{codegreen}{rgb}{0.1,0.6,0.1} % Darker green for comments

% --- Page Geometry ---
\geometry{
  a4paper,
  left=2.5cm,
  right=2.5cm,
  top=2.5cm,
  bottom=2.5cm,
  headheight=15pt
}
% --- Header/Footer Setup ---
\pagestyle{fancy}
\fancyhf{}
\fancyhead[L]{\small Rapport de Stage - Semaine 5 - Jour 2} % Updated
\fancyhead[R]{\small Zakaria el Khaldi}
\fancyfoot[C]{\thepage}
\renewcommand{\headrulewidth}{0.4pt}
\renewcommand{\footrulewidth}{0.4pt}
% --- Title Formatting ---
\titleformat{\section}
  {\normalfont\Large\bfseries\color{primary}}
  {\thesection}{1em}{}
\titleformat{\subsection}
  {\normalfont\large\bfseries\color{secondary}}
  {\thesubsection}{1em}{}
\titleformat{\subsubsection}
  {\normalfont\normalsize\bfseries\color{accent}}
  {\thesubsubsection}{1em}{}
% --- List Formatting ---
\setlist[itemize]{leftmargin=*, nosep}
\setlist[enumerate]{leftmargin=*, nosep}
% --- Hyperlink Setup ---
\hypersetup{
  colorlinks=true,
  linkcolor=primary,
  urlcolor=secondary,
  citecolor=accent
}

% --- Listings Setup for JSON ---
\lstdefinestyle{json}{
    language=json,
    basicstyle=\ttfamily\footnotesize,
    numbers=left,
    numberstyle=\tiny\color{codegray},
    stepnumber=1,
    numbersep=5pt,
    backgroundcolor=\color{white!95!black}, % Very light gray background
    showspaces=false,
    showstringspaces=false,
    showtabs=false,
    frame=tb, % Top and bottom frame
    framextopmargin=3pt,
    framexbottommargin=3pt,
    rulecolor=\color{black!30!white},
    tabsize=2,
    captionpos=b,
    breaklines=true,
    breakatwhitespace=false,
    stringstyle=\color{codepurple},
    commentstyle=\color{codegreen},
    keywordstyle=\color{codeblue}, % For true, false, null
    morestring=[b]",
    literate=
     *{0}{{{\color{codeblue}0}}}{1}
      {1}{{{\color{codeblue}1}}}{1}
      {2}{{{\color{codeblue}2}}}{1}
      {3}{{{\color{codeblue}3}}}{1}
      {4}{{{\color{codeblue}4}}}{1}
      {5}{{{\color{codeblue}5}}}{1}
      {6}{{{\color{codeblue}6}}}{1}
      {7}{{{\color{codeblue}7}}}{1}
      {8}{{{\color{codeblue}8}}}{1}
      {9}{{{\color{codeblue}9}}}{1}
      {:}{{{\color{black}:}}}{1}
      {\{}{{{\color{black}{\{}}}}{1}
      {\}}{{{\color{black}{\}}}}}{1}
      {[}{{{\color{black}{[}}}}{1}
      {]}{{{\color{black}{]}}}}{1}
      {,}{{{\color{black}{,}}}}{1},
}


% --- Title Page Information ---
\title{\Huge\bfseries\color{primary} Rapport de Stage \\ 
      \Large Semaine 5 - Jour 2 : Conception de la Page d'Accueil - Section Hero et Recherche d'Inspiration} % Updated title
\author{\Large Zakaria el Khaldi}
\date{\large Le 4 juin 2025} % Updated date for Day 2, Week 5 (Tuesday)

% --- Document Start ---
\begin{document}
% --- Cover Page ---
\begin{titlepage}
  \centering
  \vspace*{\stretch{0.5}}
  {\Huge\bfseries\color{primary} Rapport de Stage \par}
  \vspace{1cm}
  {\Large\itshape Semaine 5 - Jour 2 : Exploration Visuelle, Prototypage de la Section "Hero" et Demande de Contexte pour la Marque IAAI\par} % Updated title
  \vspace{2cm}
  
  \vspace{2cm}
  {\Large Zakaria el Khaldi\par}
  \vfill
  {\large Le 3 juin 2025\par} % Date of activity day
  \vspace*{\stretch{1}}
\end{titlepage}

% --- Table of Contents ---
\tableofcontents
\thispagestyle{empty}
\newpage

% --- Introduction ---
\section{Introduction}
\thispagestyle{fancy}
Ce rapport détaille les activités du deuxième jour de la cinquième semaine de stage. La journée a été intensivement consacrée à la conception de la page d'accueil globale de la plateforme LearnExpert, avec un accent particulier sur la section "hero". Une exploration approfondie de diverses mises en page et effets visuels a été menée, aboutissant à la création et au déploiement d'un prototype initial. Une démarche d'intégration progressive des effets a été adoptée. Parallèlement, une demande d'informations contextuelles sur la marque IAAI a été formulée pour mieux aligner le design.

% --- Day's Accomplishments ---
\section{Activités du Jour (Mardi 3 Juin 2025)} % Updated day and date

\subsection{Exploration de Mises en Page et Collecte d'Effets Visuels}
Une part significative de la journée a été dédiée à la recherche et à l'expérimentation de concepts visuels pour la page d'accueil.
\begin{itemize}
    \item De nombreuses mises en page (layouts) potentielles ont été examinées et évaluées pour leur impact et leur adéquation avec les objectifs de la plateforme.
    \item Une collection d'effets visuels (animations, transitions, interactions) a été constituée, en vue d'une intégration potentielle pour dynamiser l'expérience utilisateur.
    \item L'objectif de cette phase exploratoire était d'élargir le champ des possibles avant de se concentrer sur des implémentations spécifiques.
\end{itemize}

\subsection{Développement et Déploiement d'un Prototype de la Section "Hero"}
Sur la base des inspirations recueillies, un premier prototype de la section "hero" de la page d'accueil a été développé.
\begin{itemize}
    \item Cette section, étant la première visible par les utilisateurs, revêt une importance cruciale pour capter l'attention et communiquer la proposition de valeur.
    \item Le prototype intègre certains des layouts et effets jugés prometteurs lors de la phase d'exploration.
    \item Ce prototype a été testé pour sa fonctionnalité de base et déployé dans un environnement de démonstration.
    \item Il est important de souligner qu'il s'agit d'une version préliminaire, destinée à servir de base pour des itérations et améliorations futures. L'objectif principal à ce stade est de collecter un maximum d'inspiration visuelle et de valider certains concepts.
\end{itemize}

\subsection{Approche Itérative pour l'Intégration des Effets}
Une méthode d'intégration prudente et progressive des effets visuels a été adoptée.
\begin{itemize}
    \item Chaque effet est ajouté et testé individuellement pour s'assurer de son bon fonctionnement et de sa compatibilité avec les autres éléments de la page.
    \item Cette approche permet d'éviter les conflits potentiels et de garantir une performance optimale, tout en s'assurant que les effets se complètent harmonieusement.
\end{itemize}

\subsection{Demande de Contexte Stratégique sur la Marque IAAI}
Afin d'optimiser le thème et le langage de conception de la page d'accueil pour qu'ils reflètent fidèlement l'identité de la marque, une demande formelle d'informations contextuelles sur IAAI a été formulée.
\begin{itemize}
    \item Il a été souligné que la réception de suggestions et de critiques sur le prototype actuel serait hautement profitable.
    \item Un document ou des informations précisant les éléments suivants seraient particulièrement utiles :
    \begin{itemize}
        \item La mission et les objectifs de IAAI.
        \item Le slogan officiel ou les messages clés de la marque.
        \item La démographie cible principale de la plateforme.
    \end{itemize}
    \item Ces informations sont essentielles pour affiner le design, qui reste actuellement relativement générique en attendant ce contexte plus précis.
\end{itemize}

\subsection{Objectif de Conception : Équilibre et Clarté}
L'objectif global de la conception actuelle de la page d'accueil est de trouver un équilibre judicieux.
\begin{itemize}
    \item Il s'agit de créer une interface utilisateur qui soit à la fois engageante grâce à des effets visuels et des "accroches" (hooks) pertinents, tout en maintenant une esthétique minimaliste.
    \item Cette approche vise à garantir que le message principal de la plateforme soit transmis de manière claire, directe et sans distraction superflue.
\end{itemize}

\subsection{Planification pour l me Jour Suivant (Mercredi 4 Juin 2025)}
Pour la journée de demain, la stratégie sera la suivante :
\begin{itemize}
  \item Poursuivre l'implémentation des autres sections de la page d'accueil, en adoptant une approche progressive, section par section.
  \item Consacrer le temps nécessaire à la recherche et à l'expérimentation pour chaque nouvelle section afin d'atteindre un design de haute qualité.
  \item Continuer à viser un équilibre entre des effets visuels engageants et une interface utilisateur minimaliste pour une communication claire.
  \item Intégrer les éventuels retours sur le prototype de la section "hero" et, si disponibles, les informations de branding concernant IAAI.
  \item Documenter les choix de conception et les inspirations retenues pour référence future.
\end{itemize}

\section{Conclusion}
Cette deuxième journée de la cinquième semaine a été productive en termes d'exploration créative et de prototypage initial pour la section "hero" de la page d'accueil. La démarche itérative et la recherche d'un équilibre entre esthétique et fonctionnalité sont au cœur du processus de design. La demande d'informations contextuelles sur la marque IAAI marque une étape importante vers la personnalisation et l'optimisation du message visuel. La prise de temps pour la recherche et l'expérimentation continue est jugée essentielle pour parvenir à un design final de premier ordre.

\end{document}