\documentclass[12pt, a4paper]{report}

% --- Packages ---
\usepackage[utf8]{inputenc}
\usepackage[T1]{fontenc}
\usepackage[french]{babel}
\usepackage{graphicx}
\usepackage{booktabs}
\usepackage{amsmath}
\usepackage{geometry}
\usepackage{array}
\usepackage{enumitem}
\usepackage{hyperref}
\usepackage{xcolor}
\usepackage{titlesec}
\usepackage{lmodern}
\usepackage{microtype}
\usepackage{fancyhdr}
\usepackage{listings}
\usepackage[scaled=0.85]{beramono}
\usepackage{tikz}
\usepackage{afterpage}
\usepackage{tcolorbox}
\usepackage{titletoc}

% --- Font Configuration ---

% --- Color Definitions ---
\definecolor{primary}{RGB}{0,102,204}
\definecolor{secondary}{RGB}{102,102,153}
\definecolor{accent}{RGB}{204,0,0}
\definecolor{darkgray}{RGB}{90,90,90}
\definecolor{lightgray}{RGB}{240,240,240}
\definecolor{codegray}{rgb}{0.5,0.5,0.5}
\definecolor{codepurple}{rgb}{0.58,0,0.82}
\definecolor{codeblue}{rgb}{0,0,0.9}
\definecolor{codegreen}{rgb}{0.1,0.6,0.1}

% --- Page Geometry ---
\geometry{
  a4paper,
  left=2.5cm,
  right=2.5cm,
  top=2.5cm,
  bottom=2.5cm,
  headheight=15pt
}

% --- Header/Footer Setup ---
\pagestyle{fancy}
\fancyhf{}
\fancyhead[L]{\small Rapport de Stage}
\fancyhead[R]{\small Zakaria el Khaldi}
\fancyfoot[C]{\thepage}
\renewcommand{\headrulewidth}{0.4pt}
\renewcommand{\footrulewidth}{0.4pt}

% --- Title Formatting ---
\titleformat{\chapter}
  {\normalfont\LARGE\bfseries\color{primary}}
  {\thechapter}{1em}{}
\titlespacing*{\chapter}{0pt}{50pt}{40pt}

\titleformat{\section}
  {\normalfont\Large\bfseries\color{primary}}
  {\thesection}{1em}{}
\titleformat{\subsection}
  {\normalfont\large\bfseries\color{secondary}}
  {\thesubsection}{1em}{}
\titleformat{\subsubsection}
  {\normalfont\normalsize\bfseries\color{accent}}
  {\thesubsubsection}{1em}{}

% --- List Formatting ---
\setlist[itemize]{leftmargin=*, nosep}
\setlist[enumerate]{leftmargin=*, nosep}

% --- Hyperlink Setup ---
\hypersetup{
  colorlinks=true,
  linkcolor=primary,
  urlcolor=secondary,
  citecolor=accent
}

% --- Listings Setup for Code ---
\lstdefinestyle{codestyle}{
    basicstyle=\ttfamily\footnotesize,
    numbers=left,
    numberstyle=\tiny\color{codegray},
    stepnumber=1,
    numbersep=5pt,
    backgroundcolor=\color{white!95!black},
    showspaces=false,
    showstringspaces=false,
    showtabs=false,
    frame=tb,
    framextopmargin=3pt,
    framexbottommargin=3pt,
    rulecolor=\color{black!30!white},
    tabsize=2,
    captionpos=b,
    breaklines=true,
    breakatwhitespace=false,
    stringstyle=\color{codepurple},
    commentstyle=\color{codegreen},
    keywordstyle=\color{codeblue}
}

% --- Image path configuration ---
% When you upload images to Overleaf, place them in folders named:
% week_1_img, week_2_img, week_3_img, week_4_img
\graphicspath{{week_1_img/}{week_2_img/}{week_3_img/}{week_4_img/}{./}}

% --- Document Information ---
\title{\Huge\bfseries\color{primary} Rapport de Stage \\ 
       \Large Conception et Développement d'une Plateforme E-learning}
\author{\Large Zakaria el Khaldi}
\date{\large Juin 2025}

% --- TOC Formatting ---
\renewcommand{\contentsname}{Sommaire}

% New TOC formatting
\titlecontents{chapter}
    [0em]                                       % left margin
    {\vspace{1.5em}\large\bfseries}             % above code (font size, weight)
    {\color{primary}}                           % numbered entry format (removed number)
    {\color{primary}}                           % unnumbered entry format
    {\titlerule*[1pc]{.}\contentspage}          % filler and page number
    [\vspace{0.5em}]                            % below code

% Format for section entries
\titlecontents{section}
    [2.3em]                               % left margin
    {\normalfont}                         % above code
    {\contentslabel{2.3em}}              % numbered entry format
    {}                                    % unnumbered entry format
    {\titlerule*[1pc]{.}\contentspage}    % filler and page number

% Format for subsection entries
\titlecontents{subsection}
    [4.6em]                               % left margin
    {\small}                              % above code
    {\contentslabel{2.8em}}              % numbered entry format
    {}                                    % unnumbered entry format
    {\titlerule*[1pc]{.}\contentspage}    % filler and page number

% Format for subsubsection entries
\titlecontents{subsubsection}
    [7.4em]                               % left margin
    {\small\itshape}                      % above code
    {\contentslabel{3.3em}}              % numbered entry format
    {}                                    % unnumbered entry format
    {\titlerule*[1pc]{.}\contentspage}    % filler and page number

% --- Chapter Styling ---
\def\@makechapterhead#1{%
  \vspace*{50\p@}%
  {\parindent \z@ \raggedright \normalfont
    \ifnum \c@secnumdepth >\m@ne
      \if@mainmatter
        \huge\bfseries\color{primary} \thechapter \\ #1
        \par\nobreak
        \vskip 20\p@
      \fi
    \fi
    \interlinepenalty\@M
    \ifnum \c@secnumdepth >\m@ne
      \if@mainmatter
      \else
        \Huge \bfseries \color{primary} #1
      \fi
    \else
      \Huge \bfseries \color{primary} #1
    \fi
    \par\nobreak
    \vskip 40\p@
    \hrule
    \vskip 40\p@
  }
  \newpage
}

\def\@makeschapterhead#1{%
  \vspace*{50\p@}%
  {\parindent \z@ \raggedright
    \normalfont
    \interlinepenalty\@M
    \Huge \bfseries \color{primary} #1\par\nobreak
    \vskip 40\p@
    \hrule
    \vskip 40\p@
  }
  \newpage
}

% --- Apply fancy style to all chapters automatically ---
\makeatletter
\renewcommand\chapter{\if@openright\cleardoublepage\else\clearpage\fi
                    \thispagestyle{fancy}%
                    \global\@topnum\z@
                    \@afterindentfalse
                    \secdef\@chapter\@schapter}
\makeatother

% --- Prevent duplicate chapter numbers ---
\renewcommand{\thechapter}{\arabic{chapter}}
\makeatletter
\renewcommand{\@seccntformat}[1]{%
  \ifcsname format#1\endcsname
    \csname format#1\endcsname
  \else
    \csname the#1\endcsname\quad
  \fi
}
\makeatother

% Format chapter numbering (hide in document text since it's in the title)
\makeatletter
\newcommand{\formatchapter}{}
\makeatother

\begin{document}

% --- Cover Page ---
\begin{titlepage}
    \centering
    % --- Kingdom and Ministry Text ---
    {\LARGE \textbf{Royaume du Maroc}}\\[0.2cm]
    {\small \textbf{Ministère de l'Éducation Nationale, du Préscolaire et des Sports}}\\[0.2cm]
    \rule{\linewidth}{0.4mm} \\[0.4cm]

    % --- Logos Section ---
    \begin{minipage}{0.45\textwidth}
        \flushleft
        \includegraphics[width=3.5cm]{ministere.jpg}
    \end{minipage}
    \begin{minipage}{0.45\textwidth}
        \flushright
        \includegraphics[width=5cm]{bts_logo.png}
    \end{minipage}\\[1.3cm]

    % --- Enterprise Name (IAAI) ---
    \begin{tcolorbox}[colback=primary!10, colframe=primary, width=8cm, height=2.5cm, halign=center, valign=center]
        {\fontsize{40}{48}\selectfont\textcolor{primary}{\textbf{IAAI}}}
    \end{tcolorbox}
    
    \vspace{1.3cm}
    
    % --- Report Title ---
    \rule{\linewidth}{0.4mm} \\[0.3cm]
    {\Huge \textbf{Rapport de Stage}}\\[0.3cm]
    \rule{\linewidth}{0.4mm} \\[0.7cm]
    
    % --- Project Subject ---
    {\large \textbf{Projet : Conception et Développement d'une Plateforme E-learning}}\\[1.5cm]
    
    % --- Student and Supervisor Information ---
    \begin{minipage}{0.45\textwidth}
        \flushleft
        {\large \textbf{Présenté par :}}\\[0.2cm]
        {\large \textsc{Zakaria el Khaldi}}\\[0.5cm]
        {\large \textbf{Formation :} [Formation]}\\[0.2cm]
        {\large \textbf{Établissement :} [Établissement]}
    \end{minipage}
    \hfill
    \begin{minipage}{0.45\textwidth}
        \flushright
        {\large \textbf{Sous la supervision de :}}\\[0.2cm]
        {\large [Nom du superviseur]}\\[0.2cm]
        {\large [Fonction du superviseur]}
    \end{minipage}\\[1.5cm]
    
    % --- Internship Period ---
    {\large \textbf{Période de stage :}}\\[0.2cm]
    {\large Du 5 mai 2025 au [Date de fin]}\\[1.8cm]
    
    % --- Academic Year ---
    {\large \textbf{Année Académique : [Année Académique]}}\\
    
\end{titlepage}

% --- Remerciements ---
\chapter*{Remerciements}
\addcontentsline{toc}{chapter}{Remerciements}
\thispagestyle{fancy}

Je tiens à exprimer ma profonde gratitude à toutes les personnes qui ont contribué à la réussite de ce stage et à l'élaboration de ce rapport.

Je remercie tout d'abord M. Ahmed Benali, Directeur Technique de LearnTech Solutions, pour son accueil au sein de l'entreprise, sa disponibilité et ses précieux conseils qui m'ont guidé tout au long de cette expérience professionnelle.

Mes remerciements s'adressent également à l'ensemble de l'équipe de LearnTech Solutions pour leur accueil chaleureux, leur soutien et leur partage de connaissances qui ont rendu cette expérience particulièrement enrichissante.

Je suis également reconnaissant envers mon université et mes professeurs pour la qualité de l'enseignement reçu qui m'a permis d'aborder ce stage avec les compétences nécessaires.

Enfin, je remercie ma famille et mes amis pour leur soutien inconditionnel tout au long de mon parcours.

\newpage

% --- Table of Contents ---
\tableofcontents
\thispagestyle{fancy}
\newpage

% --- Include Chapter Files ---
% Each chapter.s content is in a separate file for better organization

% Create unnumbered introduction
\chapter*{Introduction}
\addcontentsline{toc}{chapter}{Introduction}
\thispagestyle{fancy}

Ce rapport présente le travail réalisé lors de mon stage de fin d'études effectué au sein de LearnTech Solutions du 10 mai au 10 juin 2025. Durant cette période, j'ai participé à la conception et au développement d'une plateforme e-learning complète destinée aux entreprises et aux professionnels individuels.

L'objectif de ce stage était de mettre en pratique les connaissances théoriques acquises durant ma formation et de les approfondir dans un contexte professionnel. Ce projet m'a permis de développer des compétences en développement web, en conception d'architecture logicielle, en gestion de bases de données et en traitement de données.

Ce rapport détaille l'ensemble des travaux réalisés durant les différentes semaines du stage, les méthodologies employées, les difficultés rencontrées et les solutions apportées, ainsi que les résultats obtenus.

% Chapter 1: Presentation of the Company and Internship Framework
\chapter*{Chapitre 1 : Présentation de l'Organisme d'Accueil}
\addcontentsline{toc}{chapter}{Chapitre 1 : Présentation de l'Organisme d'Accueil}
\thispagestyle{fancy}
\setcounter{section}{0}
\newpage

\section{Introduction}

Dans le cadre de mon stage de fin d'études au sein de l'International Academy of Artificial Intelligence (IAAI Academy) à Casablanca, ce chapitre présente en détail l'organisme d'accueil. Il définit le contexte institutionnel, pédagogique et technologique dans lequel mon immersion professionnelle s'est déroulée, offrant une vue d'ensemble indispensable pour appréhender les missions et compétences mobilisées.

Ce chapitre permettra de :
\begin{itemize}
  \item Identifier l'origine et la mission de l'IAAI Academy
  \item Décrire sa structure organisationnelle et ses ressources matérielles et humaines
  \item Mettre en lumière son positionnement innovant à l'intersection de l'intelligence artificielle, du développement Web et mobile, et de l'éducation
\end{itemize}

\section{Historique et Genèse de l'IAAI Academy}

\begin{itemize}
  \item \textbf{Date de création :} Fondée en 2024 à Casablanca par un collectif de docteurs et d'experts en intelligence artificielle, animés par la volonté de transformer le paysage éducatif marocain.
  
  \item \textbf{Contexte et motivations :} Face à l'évolution rapide du secteur de l'IA et aux défis du XXIᵉ siècle, l'IAAI Academy se donne pour objectif de :
  \begin{itemize}
    \item Renforcer les compétences nationales en IA
    \item Réinventer la pédagogie grâce à des solutions assistées par IA
    \item Devenir un pôle de recherche appliquée et d'innovation au service des étudiants, des enseignants et des entreprises
  \end{itemize}
\end{itemize}

\section{Mission, Vision et Valeurs}

\begin{figure}[H]
  \centering
  \includegraphics[width=0.8\textwidth]{images/mession.png}
  \caption{\textbf{Mission et vision de l'IAAI Academy}}
  \label{fig:mission}
\end{figure}

\begin{itemize}
  \item \textbf{Mission :} Former des professionnels capables de concevoir, déployer et piloter des solutions d'intelligence artificielle à fort impact, en s'appuyant sur une pédagogie innovante et des outils technologiques avancés.
  
  \item \textbf{Vision :} Devenir la référence au Maroc et en Afrique pour l'enseignement supérieur, l'innovation pédagogique et la recherche appliquée en IA, tout en contribuant à la transformation digitale du secteur éducatif.
  
  \item \textbf{Valeurs :}
  \begin{itemize}
    \item Excellence académique et exigence professionnelle
    \item Innovation pédagogique via l'intégration des technologies IA
    \item Éthique et esprit collaboratif
    \item Agilité et adaptabilité aux évolutions technologiques
  \end{itemize}
\end{itemize}

\section{Structure Organisationnelle et Gouvernance}

\begin{figure}[H]
  \centering
  \includegraphics[width=0.9\textwidth]{images/Structure Organisationnelle et Gouvernance.png}
  \caption{\textbf{Organigramme et structure de gouvernance de l'IAAI Academy}}
  \label{fig:structure}
\end{figure}

L'IAAI Academy dispose d'une structure organisationnelle efficace, composée de plusieurs départements et unités qui travaillent en synergie pour atteindre les objectifs de l'institution. La gouvernance est assurée par un conseil d'administration et un comité exécutif qui définissent les orientations stratégiques et supervisent la mise en œuvre des projets.

\section{Offres de Formation et Domaines d'Intervention}

\begin{figure}[H]
  \centering
  \includegraphics[width=0.9\textwidth]{images/Offres de Formation et Domaines d'Intervention.png}
  \caption{\textbf{Panorama des formations et domaines d'expertise de l'IAAI Academy}}
  \label{fig:formations}
\end{figure}

\begin{itemize}
  \item \textbf{Formations professionnelles en IA et Data Science :}
  \begin{itemize}
    \item Modules intensifs (Python, Machine Learning, Deep Learning, NLP, vision par ordinateur)
    \item Certifications reconnues (IBM, Microsoft, Google, Coursera)
  \end{itemize}
  
  \item \textbf{Préparation aux concours post-bac :}
  \begin{itemize}
    \item Parcours pour médecine, pharmacie, ENSA/ENSAM, ENCG
    \item Formats présentiel, à distance ou hybride, avec examens blancs et coaching
  \end{itemize}
  
  \item \textbf{Soutien scolaire individualisé :}
  \begin{itemize}
    \item Cours sur mesure en mathématiques, physique-chimie, SVT et informatique
    \item Suivi des progrès, méthodes de travail et préparation aux examens
  \end{itemize}
  
  \item \textbf{Consulting et prestations IA :}
  \begin{itemize}
    \item Développement de chatbots, automatisation de processus et analyses prédictives
    \item Accompagnement à la transformation numérique
  \end{itemize}
  
  \item \textbf{Clubs technologiques et ateliers jeunesse :}
  \begin{itemize}
    \item Initiation à la programmation, robotique éducative et réalité augmentée
    \item Projets ludiques pour stimuler l'innovation dès le plus jeune âge
  \end{itemize}
\end{itemize}

\section{Pédagogie et Méthodes d'Apprentissage}

\begin{figure}[H]
  \centering
  \includegraphics[width=0.9\textwidth]{images/Pédagogie et Méthodes d'Apprentissage.png}
  \caption{\textbf{Approche pédagogique innovante de l'IAAI Academy}}
  \label{fig:pedagogie}
\end{figure}

L'IAAI Academy adopte une approche pédagogique innovante, intégrant les technologies d'intelligence artificielle pour personnaliser l'apprentissage et optimiser l'acquisition des compétences. Cette méthode combine enseignement théorique, pratique intensive et projets concrets, enrichis par des outils d'évaluation continue et d'adaptation du contenu au profil de chaque apprenant.

\section{Impact de mon Stage}

Ce stage m'a permis de :
\begin{itemize}
  \item M'immerger dans un environnement professionnel concret, en participant à des projets IA et pédagogiques
  \item Renforcer mes compétences en data science et technologies IA, ainsi qu'en développement Web (React) et mobile (React Native)
  \item Collaborer au sein d'une équipe pluridisciplinaire, développant ma capacité d'adaptation et de communication
  \item Adopter une posture critique et réflexive, afin de proposer des solutions innovantes et répondre aux enjeux réels du terrain
\end{itemize}

\section{Conclusion}

L'IAAI Academy a offert un cadre d'apprentissage et de recherche exemplaire, conjuguant ressources innovantes, démarche pédagogique assistée par IA et infrastructures technologiques avancées. Ce stage m'a non seulement permis de consolider mes compétences et de comprendre les enjeux de l'intégration de l'IA en contexte réel, mais il m'a également ouvert de nouvelles perspectives professionnelles et académiques dans le domaine de l'intelligence artificielle appliquée à l'éducation. 

% Chapter 2: Requirements and Specifications
\chapter{Chapitre 2 : Cahier des Charges et Spécifications}
\thispagestyle{fancy}
\newpage

Ce chapitre présente le projet LearnExpert, ses objectifs, l'analyse des besoins fonctionnels et non fonctionnels, ainsi que les technologies choisies pour sa réalisation.

\section{Présentation du Projet LearnExpert}

\subsection{Contexte et Problématique}
Le secteur de l'éducation en ligne connaît une croissance exponentielle, accélérée encore davantage par la pandémie mondiale qui a révélé les limites des plateformes d'apprentissage existantes. Dans ce contexte, plusieurs problématiques ont été identifiées :

\begin{itemize}
  \item Les plateformes actuelles proposent souvent un contenu générique et peu personnalisé
  \item Les méthodes d'apprentissage de la programmation restent trop théoriques, avec un manque d'interactivité
  \item Les apprenants font face à une fragmentation des ressources éducatives, nécessitant de naviguer entre différentes plateformes
  \item Les outils d'apprentissage ne s'adaptent pas au rythme et au niveau de chaque utilisateur
\end{itemize}

Face à ces défis, IAAI a décidé de développer LearnExpert, une plateforme innovante centrée sur l'apprentissage de la programmation et des technologies web.

\subsection{Objectifs du Projet}
Le projet LearnExpert vise à créer une plateforme d'apprentissage complète qui se distingue par :

\begin{itemize}
  \item Une approche centrée sur la pratique et l'interactivité pour l'apprentissage de la programmation
  \item Une personnalisation de l'expérience d'apprentissage grâce à l'IA
  \item Une intégration de contenus structurés et de haute qualité pour divers langages de programmation
  \item Une architecture évolutive permettant d'ajouter de nouvelles fonctionnalités et de nouveaux contenus
  \item Une expérience utilisateur fluide et engageante
\end{itemize}

L'objectif principal est de créer une plateforme qui accompagne efficacement les apprenants dans leur parcours d'acquisition de compétences techniques, qu'ils soient débutants ou développeurs expérimentés cherchant à se perfectionner.

\section{Analyse des Besoins}

\subsection{Besoins Fonctionnels}
Les fonctionnalités clés identifiées pour la plateforme LearnExpert sont :

\begin{itemize}
  \item \textbf{Gestion des utilisateurs et authentification}
    \begin{itemize}
      \item Inscription et connexion des utilisateurs
      \item Profils utilisateurs avec suivi de progression
      \item Système de rôles (apprenant, créateur de contenu, administrateur)
    \end{itemize}
  
  \item \textbf{Gestion des cours et contenus}
    \begin{itemize}
      \item Catalogue de cours organisé par catégories et niveaux
      \item Système de modules et de leçons structurés
      \item Support pour divers formats de contenu (texte, code, vidéo)
    \end{itemize}
  
  \item \textbf{Apprentissage interactif}
    \begin{itemize}
      \item Éditeur de code intégré avec exécution en temps réel
      \item Exercices pratiques et projets guidés
      \item Tests automatisés pour valider les compétences
    \end{itemize}
  
  \item \textbf{Analyse et suivi}
    \begin{itemize}
      \item Tableau de bord de progression pour les apprenants
      \item Statistiques d'utilisation pour les administrateurs
      \item Recommandations personnalisées basées sur les performances
    \end{itemize}
  
  \item \textbf{Interface d'administration}
    \begin{itemize}
      \item Gestion des utilisateurs et des permissions
      \item Création et édition de contenu éducatif
      \item Suivi des métriques de la plateforme
    \end{itemize}
\end{itemize}

\subsection{Besoins Non Fonctionnels}
Au-delà des fonctionnalités, plusieurs exigences non fonctionnelles ont été définies :

\begin{itemize}
  \item \textbf{Performance}
    \begin{itemize}
      \item Temps de chargement des pages inférieur à 2 secondes
      \item Capacité à supporter au moins 1000 utilisateurs simultanés
      \item Exécution rapide du code dans l'environnement intégré
    \end{itemize}
  
  \item \textbf{Sécurité}
    \begin{itemize}
      \item Protection des données personnelles des utilisateurs
      \item Isolation des environnements d'exécution de code
      \item Authentification robuste et gestion sécurisée des sessions
    \end{itemize}
  
  \item \textbf{Accessibilité et UX}
    \begin{itemize}
      \item Interface intuitive et responsive (mobile, tablette, desktop)
      \item Respect des standards d'accessibilité WCAG 2.1
      \item Support multilingue (initialement français et anglais)
    \end{itemize}
  
  \item \textbf{Évolutivité et maintenance}
    \begin{itemize}
      \item Architecture modulaire facilitant l'ajout de nouvelles fonctionnalités
      \item Documentation technique complète
      \item Tests automatisés couvrant les fonctionnalités critiques
    \end{itemize}
\end{itemize}

\section{Technologies et Outils}

\subsection{Stack Technologique}
Après analyse des besoins et des contraintes du projet, les technologies suivantes ont été sélectionnées :

\begin{itemize}
  \item \textbf{Frontend}
    \begin{itemize}
      \item Next.js (framework React) pour le développement d'une application web moderne
      \item TypeScript pour garantir la robustesse et la maintenabilité du code
      \item Tailwind CSS pour un design responsive et cohérent
      \item Monaco Editor pour l'implémentation de l'éditeur de code interactif
      \item Framer Motion pour les animations et transitions fluides
    \end{itemize}
  
  \item \textbf{Backend}
    \begin{itemize}
      \item Architecture microservices avec Node.js et Express
      \item NGINX comme API Gateway pour la gestion des requêtes
      \item Supabase pour l'authentification et les services de base de données
      \item MongoDB pour le stockage des contenus de cours et des données structurées
    \end{itemize}
  
  \item \textbf{Traitement des données}
    \begin{itemize}
      \item Python pour les scripts de scraping et de nettoyage de données
      \item APIs LLM (Large Language Models) pour le traitement intelligent des contenus
      \item JSON comme format principal pour le stockage et l'échange de données
    \end{itemize}
  
  \item \textbf{Déploiement et infrastructure}
    \begin{itemize}
      \item Vercel pour l'hébergement du frontend
      \item Docker pour la conteneurisation des services backend
      \item GitHub Actions pour l'intégration et le déploiement continus
    \end{itemize}
\end{itemize}

\subsection{Environnement de Développement}
L'environnement de développement a été configuré pour optimiser la productivité et la collaboration :

\begin{itemize}
  \item Visual Studio Code comme IDE principal
  \item ESLint et Prettier pour le linting et le formatage du code
  \item Git et GitHub pour le contrôle de version et la collaboration
  \item Postman pour le test des API
  \item Figma pour la conception et le prototypage de l'interface utilisateur
\end{itemize}

Cette stack technologique a été choisie pour sa modernité, sa flexibilité et sa capacité à répondre aux besoins spécifiques du projet, tout en garantissant des performances optimales et une bonne expérience utilisateur. 

% Chapter 3: Platform Design and Architecture
\chapter*{Chapitre 3 : Conception et Modélisation}
\addcontentsline{toc}{chapter}{Chapitre 3 : Conception et Modélisation}
\thispagestyle{fancy}
\setcounter{section}{0}
\newpage

\section{Introduction}
La première phase du projet a été consacrée à la conception initiale, à la définition de l'architecture backend et des bases de données, ainsi qu'à la création des premiers diagrammes UML.

\section{Méthodologie Adoptée}
La méthodologie adoptée pour la conception du système s'est basée sur une approche structurée et itérative.

\subsection{Modèle en cascade}
Pour la phase de conception, nous avons adopté une approche en cascade adaptée, permettant de définir clairement les étapes successives tout en maintenant la possibilité de réviser les décisions précédentes au besoin.

\subsection{Langage UML}
Le langage UML (Unified Modeling Language) a été utilisé comme standard pour représenter graphiquement l'architecture et les interactions du système. Ce choix permet une communication claire et non ambiguë entre toutes les parties prenantes du projet.

\section{Conception}

\subsection{Identification des acteurs}
Les principaux types d'utilisateurs identifiés pour la plateforme sont :
\begin{itemize}
  \item \textbf{Apprenant Individuel :} S'inscrit, s'abonne, apprend, demande des consultations
  \item \textbf{Employé d'Entreprise :} Apprend via l'abonnement de l'entreprise, demande des consultations
  \item \textbf{Administrateur/Manager d'Entreprise :} Gère le compte de l'entreprise, les employés, les abonnements, attribue les cours, consulte les analyses
  \item \textbf{Créateur de Cours :} Conçoit et élabore le contenu des cours (modules, leçons, quiz)
  \item \textbf{Consultant/Fournisseur de Prestations :} Gère sa disponibilité, anime les sessions via la plateforme de réunion
  \item \textbf{Agent de Support Plateforme :} Assiste les utilisateurs pour les problèmes liés à la plateforme et les questions-réponses
  \item \textbf{Administrateur de la Plateforme :} Supervise l'ensemble de la plateforme, les utilisateurs, le contenu, les paramètres
\end{itemize}

\subsection{Identification des messages}
La plateforme e-learning conçue repose sur les fonctionnalités essentielles suivantes :
\begin{itemize}
    \item \textbf{E-Learning :}
    \begin{itemize}
        \item Catalogue de cours (filtrable, consultable)
        \item Modules, Leçons (vidéo, basées sur des images)
        \item Quiz et Évaluations
        \item Suivi de la Progression
        \item Certificats de Réussite
        \item Support Multilingue (EN/FR)
        \item Paramètres Utilisateur et Entreprise
    \end{itemize}
    \item \textbf{Consultation et Prestation :}
    \begin{itemize}
        \item Navigation des services
        \item Profils des consultants et disponibilité
        \item Système de réservation/demande
        \item Intégration API avec la plateforme de réunion personnalisée
        \item Facturation des sessions
    \end{itemize}
    \item \textbf{Monétisation :}
    \begin{itemize}
        \item Abonnements pour utilisateurs individuels (accès illimité)
        \item Abonnements pour entreprises (par utilisateur)
        \item Réductions et Démos
    \end{itemize}
\end{itemize}

\subsection{Les diagrammes de cas d'utilisation}
Les diagrammes de cas d'utilisation ont permis de représenter visuellement les interactions entre les acteurs et le système.

\begin{figure}[p]
  \centering
  \includegraphics[width=0.52\textwidth,keepaspectratio]{images/usecase_diagram.png}
  \caption{\textbf{Diagramme de cas d'utilisation} montrant les principales fonctionnalités accessibles par les acteurs.}
  \label{fig:usecase_diagram}
\end{figure}
\clearpage

\subsection{Les diagrammes de séquence}
Les diagrammes de séquence ont permis d'illustrer les interactions temporelles entre les différents composants du système pour des scénarios clés.

Pour illustrer le fonctionnement de l'architecture, voici un exemple de flux de communication entre les services pour un scénario d'inscription et d'abonnement :
\begin{enumerate}
  \item \textbf{Client (Next.js)} $\rightarrow$ \textbf{Passerelle API (Nginx)} $\rightarrow$ \textbf{Service IAM} (Création de l'utilisateur)
  \item \textbf{Service IAM} $\rightarrow$ \textbf{Kafka} (Publie `UserRegisteredEvent`)
  \item \textbf{Service de Notification} (Consomme l'événement) $\rightarrow$ Envoie un Email de Bienvenue
  \item \textbf{Client} $\rightarrow$ \textbf{Passerelle API} $\rightarrow$ \textbf{Service de Facturation} (Demande d'abonnement)
  \item \textbf{Service de Facturation} $\rightarrow$ Passerelle de Paiement et Mise à jour de la BD interne
  \item \textbf{Service de Facturation} $\rightarrow$ \textbf{Kafka} (Publie `SubscriptionActivatedEvent`)
  \item \textbf{Service IAM} (Consomme, met à jour le statut utilisateur) \& \textbf{Service de Notification} (Consomme, envoie une confirmation)
\end{enumerate}

\begin{figure}[p]
  \centering
  \includegraphics[width=0.9\textwidth,keepaspectratio]{images/sequence_diagram.png}
  \caption{\textbf{Diagramme de séquence} illustrant le processus d'authentification et de notification.}
  \label{fig:sequence_diagram}
\end{figure}
\clearpage

\subsection{Le diagramme de classes}
Le diagramme de classes a été élaboré pour représenter les principales entités du système et leurs relations.

\begin{figure}[p]
  \centering
  \includegraphics[width=0.9\textwidth,keepaspectratio]{week_1_img/class_diagrame.png}
  \caption{\textbf{Diagramme de classes} de la plateforme e-learning montrant les principales entités et leurs relations.}
  \label{fig:class_diagram}
\end{figure}

Ce diagramme illustre les relations entre les différentes entités du système, comme les utilisateurs, les cours, les modules, les leçons, les abonnements et les services de consultation. Les cardinalités et les types de relations (composition, agrégation, héritage) ont été définies pour refléter précisément la structure du modèle de données.
\clearpage

\section{Architecture Adoptée}

L'architecture envisagée pour la plateforme repose sur une approche de microservices, justifiée par plusieurs avantages clés :
\begin{itemize}
  \item \textbf{Scalabilité :} Mise à l'échelle indépendante des services individuels selon les besoins
  \item \textbf{Maintenabilité :} Possibilité de modifier des parties spécifiques sans impacter l'ensemble du système
  \item \textbf{Résilience :} Limitation de l'impact des défaillances à des services spécifiques
  \item \textbf{Autonomie des Équipes :} Développement, tests et déploiement indépendants par différentes équipes
  \item \textbf{Flexibilité Technologique :} Utilisation des technologies les plus adaptées pour chaque service
\end{itemize}

\subsection{Pile Technologique}
La pile technologique définie pour le développement comprend :
\begin{itemize}
  \item \textbf{Frontend :} Next.js (React)
  \item \textbf{Microservices Backend :}
    \begin{itemize}
      \item Go (pour les services critiques en performance et concurrents comme les Notifications, le backend de la plateforme de réunion)
      \item Python avec FastAPI (pour les services gourmands en données, développement rapide d'API, par exemple, Catalogue de Cours, Facturation)
      \item Node.js avec Express (TypeScript) (pour les opérations I/O intensives, interaction Supabase, par exemple, IAM, Gestion des Médias)
    \end{itemize}
  \item \textbf{Bases de Données :}
    \begin{itemize}
      \item PostgreSQL (stockage relationnel principal pour la plupart des services)
      \item Supabase (pour l'Authentification, le Stockage, et son Postgres géré pour des services spécifiques)
    \end{itemize}
  \item \textbf{Passerelle API :} Nginx (en tant que reverse proxy et passerelle)
  \item \textbf{Broker de Messages :} Apache Kafka (pour une gestion d'événements asynchrones robuste et scalable)
  \item \textbf{Conteneurisation et Orchestration :} Docker (Kubernetes serait une étape logique suivante pour l'orchestration)
\end{itemize}

\subsection{Modèles de Données des Services}
Dans le cadre de cette première phase, des modèles de données préliminaires ont été conçus pour chacun des services identifiés. Voici quelques exemples des structures de données principales :

\subsubsection{Service IAM (Identity and Access Management)}
\begin{figure}[p]
  \centering
  \includegraphics[width=0.8\textwidth,keepaspectratio]{week_1_img/services_db_screanshots/Screenshot 2025-06-06 at 15-08-36 IAM_Service.pdf.png}
  \caption{\textbf{Modèle de données du service IAM} pour la gestion des utilisateurs et des autorisations.}
  \label{fig:iam_service}
\end{figure}

\vspace{5pt}
\small
\paragraph{Points clés du modèle IAM :}
\begin{itemize}[leftmargin=*,noitemsep,topsep=0pt]
  \item \textbf{Gestion centralisée des utilisateurs} avec tables dédiées aux utilisateurs, rôles et permissions
  \item \textbf{Support multi-tenant} via la table des entreprises (companies)
  \item \textbf{Système de référence} permettant le suivi des recommandations et affiliations
  \item \textbf{Paramètres utilisateurs} stockés de manière structurée pour personnaliser l'expérience
  \item \textbf{Jetons d'authentification} permettant une gestion sécurisée des sessions
\end{itemize}
\normalsize
\clearpage

\subsubsection{Service de Contenu}
\begin{figure}[p]
  \centering
  \includegraphics[width=0.8\textwidth,keepaspectratio]{week_1_img/services_db_screanshots/Screenshot 2025-06-06 at 15-07-51 Content_Service.pdf.png}
  \caption{\textbf{Modèle de données du service de contenu} pour la gestion des cours et ressources pédagogiques.}
  \label{fig:content_service}
\end{figure}

\vspace{5pt}
\small
\paragraph{Points clés du service de Contenu :}
\begin{itemize}[leftmargin=*,noitemsep,topsep=0pt]
  \item \textbf{Structure hiérarchique} des cours, modules et leçons
  \item \textbf{Support multimédia} avec gestion des vidéos, documents et quizz
  \item \textbf{Métadonnées riches} pour faciliter la recherche et la catégorisation
  \item \textbf{Gestion des versions} permettant la mise à jour du contenu sans perte d'historique
  \item \textbf{Support multilingue} pour internationaliser le contenu pédagogique
\end{itemize}
\normalsize
\clearpage

\subsubsection{Service de Facturation et d'Abonnement}
\begin{figure}[p]
  \centering
  \includegraphics[width=0.8\textwidth,keepaspectratio]{week_1_img/services_db_screanshots/Screenshot 2025-06-06 at 15-05-28 Billing_and_Subscription_Service.pdf.png}
  \caption{\textbf{Modèle de données du service de facturation} pour la gestion des abonnements et des paiements.}
  \label{fig:billing_service}
\end{figure}

\vspace{5pt}
\small
\paragraph{Points clés du service de Facturation :}
\begin{itemize}[leftmargin=*,noitemsep,topsep=0pt]
  \item \textbf{Gestion des plans d'abonnement} avec différents niveaux de service
  \item \textbf{Suivi des factures et paiements} pour les utilisateurs individuels et entreprises
  \item \textbf{Support des promotions et réductions} temporaires ou permanentes
  \item \textbf{Historique de facturation} complet pour analyses financières
  \item \textbf{Intégration} avec les passerelles de paiement externes
\end{itemize}
\normalsize
\clearpage

\subsubsection{Service de Certification}
\begin{figure}[p]
  \centering
  \includegraphics[width=0.8\textwidth,keepaspectratio]{week_1_img/services_db_screanshots/Screenshot 2025-06-06 at 15-05-53 Certification_Service.pdf.png}
  \caption{\textbf{Modèle de données du service de certification} pour la délivrance et la validation des certificats.}
  \label{fig:certification_service}
\end{figure}

\vspace{5pt}
\small
\paragraph{Points clés du service de Certification :}
\begin{itemize}[leftmargin=*,noitemsep,topsep=0pt]
  \item \textbf{Création de certificats} à l'achèvement des cours et modules
  \item \textbf{Validation et vérification} des compétences acquises
  \item \textbf{Badges et récompenses} pour motiver les apprenants
  \item \textbf{Système de validation externe} permettant aux entreprises de vérifier l'authenticité
  \item \textbf{Historique des certifications} pour chaque utilisateur
\end{itemize}
\normalsize
\clearpage

\subsubsection{Service d'Analytique et Reporting}
\begin{figure}[p]
  \centering
  \includegraphics[width=0.8\textwidth,keepaspectratio]{week_1_img/services_db_screanshots/Screenshot 2025-06-06 at 15-04-59 Analytics_and_Reporting_Service.pdf.png}
  \caption{\textbf{Modèle de données du service d'analytique} pour le suivi des performances et la génération de rapports.}
  \label{fig:analytics_service}
\end{figure}

\vspace{5pt}
\small
\paragraph{Points clés du service d'Analytique :}
\begin{itemize}[leftmargin=*,noitemsep,topsep=0pt]
  \item \textbf{Collecte de données d'utilisation} sur toutes les interactions utilisateurs
  \item \textbf{Métriques de performance} pour les cours et modules
  \item \textbf{Rapports personnalisés} pour les administrateurs et entreprises
  \item \textbf{Tableaux de bord en temps réel} pour le suivi des indicateurs clés
  \item \textbf{Système d'alerte} pour identifier les anomalies ou opportunités
\end{itemize}
\normalsize
\clearpage

\subsubsection{Service de Feedback}
\begin{figure}[p]
  \centering
  \includegraphics[width=0.8\textwidth,keepaspectratio]{week_1_img/services_db_screanshots/Screenshot 2025-06-06 at 15-08-00 Feedback_Service.pdf.png}
  \caption{\textbf{Modèle de données du service de feedback} pour la collecte et la gestion des retours utilisateurs.}
  \label{fig:feedback_service}
\end{figure}

\vspace{5pt}
\small
\paragraph{Points clés du service de Feedback :}
\begin{itemize}[leftmargin=*,noitemsep,topsep=0pt]
  \item \textbf{Évaluations et avis} sur les cours et modules
  \item \textbf{Système de notation} permettant une évaluation quantitative
  \item \textbf{Commentaires et suggestions} pour améliorer le contenu
  \item \textbf{Analyse de sentiment} sur les retours textuels
  \item \textbf{Gestion des signalements} pour modérer les contenus inappropriés
\end{itemize}
\normalsize
\clearpage

\subsubsection{Service de Notification}
\begin{figure}[p]
  \centering
  \includegraphics[width=0.8\textwidth,keepaspectratio]{week_1_img/services_db_screanshots/Screenshot 2025-06-06 at 15-08-13 Notification_Service.pdf.png}
  \caption{\textbf{Modèle de données du service de notification} pour la gestion des alertes et communications.}
  \label{fig:notification_service}
\end{figure}

\vspace{5pt}
\small
\paragraph{Points clés du service de Notification :}
\begin{itemize}[leftmargin=*,noitemsep,topsep=0pt]
  \item \textbf{Types de notifications} variés (email, in-app, SMS, push)
  \item \textbf{Modèles personnalisables} pour chaque type de message
  \item \textbf{Suivi de statut} (envoyé, livré, lu)
  \item \textbf{Planification} et envoi différé
  \item \textbf{Préférences utilisateurs} pour le contrôle des communications reçues
\end{itemize}
\normalsize
\clearpage

\subsubsection{Service de Gestion des Médias}
\begin{figure}[p]
  \centering
  \includegraphics[width=0.8\textwidth,keepaspectratio]{week_1_img/services_db_screanshots/Screenshot 2025-06-06 at 15-08-27 Media_Management_Service.pdf.png}
  \caption{\textbf{Modèle de données du service de gestion des médias} pour le stockage et la manipulation des ressources multimédia.}
  \label{fig:media_service}
\end{figure}

\vspace{5pt}
\small
\paragraph{Points clés du service de Gestion des Médias :}
\begin{itemize}[leftmargin=*,noitemsep,topsep=0pt]
  \item \textbf{Organisation hiérarchique} des ressources multimédia
  \item \textbf{Métadonnées détaillées} pour faciliter la recherche et l'indexation
  \item \textbf{Versionnement} des fichiers pour garder l'historique des modifications
  \item \textbf{Gestion des transformations} (redimensionnement, compression, conversion)
  \item \textbf{Contrôles d'accès} pour sécuriser les ressources sensibles
\end{itemize}
\normalsize
\clearpage

\subsubsection{Service de Configuration de la Plateforme}
\begin{figure}[p]
  \centering
  \includegraphics[width=0.8\textwidth,keepaspectratio]{week_1_img/services_db_screanshots/Screenshot 2025-06-06 at 15-08-44 Platform_Configuration_Service.pdf.png}
  \caption{\textbf{Modèle de données du service de configuration} pour la personnalisation et la gestion des paramètres de la plateforme.}
  \label{fig:config_service}
\end{figure}

\vspace{5pt}
\small
\paragraph{Points clés du service de Configuration :}
\begin{itemize}[leftmargin=*,noitemsep,topsep=0pt]
  \item \textbf{Paramètres système} pour contrôler le comportement de la plateforme
  \item \textbf{Personnalisation visuelle} (thèmes, couleurs, polices)
  \item \textbf{Fonctionnalités modulaires} activables/désactivables
  \item \textbf{Configuration multi-environnement} (dev, staging, production)
  \item \textbf{Historique des modifications} pour le suivi des changements
\end{itemize}
\normalsize
\clearpage

\subsubsection{Service de Recherche}
\begin{figure}[p]
  \centering
  \includegraphics[width=0.8\textwidth,keepaspectratio]{week_1_img/services_db_screanshots/Screenshot 2025-06-06 at 15-08-52 Search_Service.pdf.png}
  \caption{\textbf{Modèle de données du service de recherche} pour l'indexation et la recherche de contenu.}
  \label{fig:search_service}
\end{figure}

\vspace{5pt}
\small
\paragraph{Points clés du service de Recherche :}
\begin{itemize}[leftmargin=*,noitemsep,topsep=0pt]
  \item \textbf{Indexation avancée} du contenu pour des recherches rapides
  \item \textbf{Filtres et facettes} pour affiner les résultats
  \item \textbf{Recherche full-text} avec correction orthographique
  \item \textbf{Optimisation sémantique} pour comprendre l'intention de recherche
  \item \textbf{Historique et suggestions} pour améliorer l'expérience utilisateur
\end{itemize}
\normalsize
\clearpage

\begin{table}[h!]
\centering
\small
\caption{Comparaison des caractéristiques techniques des services}
\label{tab:comparaison_services}
\begin{tabular}{|l|c|c|c|}
\hline
\textbf{Service} & \textbf{Technologie principale} & \textbf{Type de données} & \textbf{Complexité} \\
\hline
IAM & Node.js/Go & Utilisateurs & Élevée \\
Contenu & Python/FastAPI & Éducation & Moyenne \\
Facturation & Python/FastAPI & Financières & Élevée \\
Certification & Python/Go & Validation & Moyenne \\
Analytique & Python & Statistiques & Élevée \\
Feedback & Node.js & Évaluation & Basse \\
Notification & Go & Messaging & Moyenne \\
Médias & Node.js & Binaires & Élevée \\
Configuration & Go & Paramètres & Basse \\
Recherche & Python/Elasticsearch & Index & Élevée \\
\hline
\end{tabular}
\end{table}
\normalsize 

\section{Conclusion}

La conception et la modélisation de la plateforme LearnExpert ont été réalisées en adoptant une approche méthodique et structurée. Les différents diagrammes UML (cas d'utilisation, séquence, classes) ont permis de représenter visuellement les interactions, flux et relations qui forment l'ossature du système, fournissant ainsi une base solide pour les phases de développement ultérieures.

Cette phase conceptuelle a été complétée par une planification détaillée du projet à travers les diagrammes de Gantt et PERT, permettant de structurer temporellement le développement et d'identifier les dépendances critiques entre les tâches.

L'architecture microservices retenue offre plusieurs avantages stratégiques pour ce projet :
\begin{itemize}
  \item Une meilleure modularité, permettant l'évolution indépendante de chaque composant
  \item Une scalabilité granulaire, adaptée aux besoins variables de chaque service
  \item Une flexibilité technologique, avec l'utilisation des langages et frameworks les plus adaptés à chaque fonctionnalité
  \item Une robustesse accrue, grâce à l'isolation des défaillances potentielles
  \item Un développement parallèle facilité, permettant à différentes équipes de travailler simultanément
\end{itemize}

La modélisation détaillée des données pour chaque service constitue un élément clé de cette phase, avec une attention particulière portée à la cohérence, l'intégrité et l'efficacité des structures de données. La documentation minutieuse des schémas de base de données, des relations entre entités et des règles métier facilitera grandement la phase d'implémentation.

Cette étape de conception a également permis d'anticiper les défis techniques potentiels et de prévoir des solutions adaptées, notamment concernant la communication inter-services, la gestion des transactions distribuées et la sécurité des données. L'approche événementielle choisie, avec l'utilisation de Kafka comme broker de messages, offre un modèle de communication asynchrone particulièrement adapté à une architecture distribuée.

En conclusion, cette phase de conception et de planification a posé les fondations architecturales, conceptuelles et organisationnelles nécessaires au développement d'une plateforme e-learning moderne, évolutive et performante, alignée avec les objectifs définis dans le cahier des charges. La visualisation claire du planning et des dépendances entre tâches via les diagrammes PERT et Gantt constitue un atout majeur pour la bonne exécution du projet dans les délais impartis.

\section{Planification et Gestion du Projet}

La planification du projet LearnExpert a été réalisée selon une approche méthodique, permettant de visualiser les différentes phases, les tâches associées, leurs durées et leurs interdépendances. Deux outils de visualisation complémentaires ont été utilisés pour représenter le planning du projet : le diagramme de Gantt et le diagramme PERT.

\subsection{Diagramme de Gantt}

Le diagramme de Gantt offre une représentation chronologique des tâches, permettant de visualiser :
\begin{itemize}[leftmargin=*,noitemsep,topsep=0pt]
  \item La durée de chaque tâche
  \item Les dates de début et de fin
  \item Le chevauchement entre les différentes tâches
  \item La répartition des tâches par phase du projet
\end{itemize}

\begin{figure}[p]
  \centering
  \includegraphics[width=0.9\textwidth,keepaspectratio]{images/gestion_projet/gantt_diagram.png}
  \caption{\textbf{Diagramme de Gantt} montrant la planification et la progression du projet.}
  \label{fig:gantt_diagram}
\end{figure}

Ce diagramme permet d'identifier clairement la durée totale du projet (du 10 mai au 8 juin 2025) ainsi que les périodes d'activité intense où plusieurs tâches se déroulent en parallèle. On peut notamment constater que les phases 1 à 3 (Conception, Frontend initial, Données et MongoDB) se déroulent principalement durant la première moitié du projet, tandis que les phases 4 à 6 (UX et Authentification, Backend, Design Landing Page) occupent la seconde moitié.

\clearpage

\subsection{Diagramme PERT}
Le diagramme PERT (Program Evaluation and Review Technique) complète le diagramme de Gantt en illustrant les dépendances entre les tâches et le chemin critique du projet.

\begin{figure}[p]
  \centering
  \includegraphics[width=0.9\textwidth,keepaspectratio]{images/gestion_projet/pert_diagram.png}
  \caption{\textbf{Diagramme PERT} montrant les relations et dépendances entre les tâches du projet.}
  \label{fig:pert_diagram}
\end{figure}

L'analyse du diagramme PERT permet de dégager plusieurs observations importantes :

\begin{itemize}[leftmargin=*,noitemsep,topsep=0pt]
  \item \textbf{Organisation modulaire} : Les tâches sont regroupées en six phases distinctes, chacune avec ses propres objectifs et livrables.
  
  \item \textbf{Dépendances inter-phases} : Certaines tâches de phases ultérieures dépendent de l'achèvement de tâches de phases antérieures, ce qui souligne l'importance de respecter les jalons intermédiaires.
  
  \item \textbf{Chemin critique} : Représenté en rouge, le chemin critique identifie les tâches dont tout retard impacterait directement la date de fin du projet. Il traverse notamment les phases de conception, de développement frontend, de pages secondaires, et s'étend jusqu'aux services backend.
  
  \item \textbf{Parallélisation des tâches} : Plusieurs activités peuvent être menées en parallèle, comme le développement des composants UI et la mise en place du système de routage, optimisant ainsi l'utilisation des ressources.
\end{itemize}

\clearpage

\subsection{Avantages de la Planification Visuelle}

L'utilisation combinée des diagrammes de Gantt et PERT présente plusieurs avantages significatifs pour la gestion du projet :

\begin{itemize}[leftmargin=*,noitemsep,topsep=0pt]
  \item \textbf{Communication claire} : Facilite la compréhension du planning par toutes les parties prenantes, techniques et non techniques.
  
  \item \textbf{Identification des risques} : Permet de repérer les zones de tension potentielles dans le planning, notamment les tâches sur le chemin critique.
  
  \item \textbf{Suivi de l'avancement} : Offre un référentiel visuel pour mesurer la progression réelle par rapport au planning initial.
  
  \item \textbf{Allocation des ressources} : Aide à l'optimisation de l'attribution des ressources humaines et techniques sur les différentes tâches.
  
  \item \textbf{Anticipation des dépendances} : Met en évidence les relations entre les tâches, permettant de préparer à l'avance les transitions entre phases.
\end{itemize}

La planification détaillée réalisée pour le projet LearnExpert constitue un élément essentiel pour assurer une exécution maîtrisée et respecter les délais fixés. Elle servira de référence tout au long du développement pour suivre l'avancement et ajuster les priorités si nécessaire. 

% Chapter 4: Development and Implementation
\chapter*{Chapitre 4 : Réalisation et mise en pied du système}
\addcontentsline{toc}{chapter}{Chapitre 4 : Réalisation et mise en pied du système}
\thispagestyle{fancy}
\setcounter{section}{0}
\newpage

\section{Outils et technologies de développement}
Pour le développement du projet, plusieurs outils et technologies modernes ont été utilisés afin d'assurer la qualité, la performance et la maintenabilité du code.

\subsection{Environnement de développement}
L'environnement de développement a été configuré avec les outils suivants :
\begin{itemize}
  \item \textbf{Visual Studio code et neovim :} Comme éditeur de code principaux avec des extensions pour React, TypeScript et ESLint
  \item \textbf{GitHub :} Pour la gestion de versions et la collaboration
  \item \textbf{Docker :} Pour la conteneurisation des services backend
  \item \textbf{Postman :} Pour tester les API
  \item \textbf{Chrome DevTools :} Pour le debugging et l'optimisation des performances frontend
\end{itemize}

\subsection{Technologies frontend}
Pour le développement front-end de cette application, les technologies suivantes ont été utilisées :
\begin{itemize}
  \item \textbf{Next.js :} Framework React pour le rendu côté serveur et la génération de sites statiques
  \item \textbf{Tailwind CSS :} Framework CSS utilitaire pour un design responsive et personnalisable
  \item \textbf{Framer Motion :} Bibliothèque d'animations pour ajouter des transitions et effets visuels
  \item \textbf{React Hook Form :} Gestion des formulaires avec validation
  \item \textbf{TypeScript :} Pour un code plus robuste avec typage statique
\end{itemize}

\subsection{Technologies backend}
Pour le développement backend, les technologies suivantes ont été utilisées :
\begin{itemize}
  \item \textbf{Node.js :} Environnement d'exécution JavaScript côté serveur
  \item \textbf{Express :} Framework web pour Node.js
  \item \textbf{PostgreSQL :} Système de gestion de base de données relationnelle
  \item \textbf{Supabase :} Plateforme Backend-as-a-Service pour l'authentification et le stockage
  \item \textbf{Redis :} Pour la mise en cache et la gestion des sessions
\end{itemize}



% Chapter 5: Deployment and Optimization
\include{chapter_5}

% Chapter 6: Assessment and Future Perspectives
\chapter{Chapitre 6 : Bilan et Perspectives}
\thispagestyle{fancy}
\newpage

Ce chapitre présente un bilan du projet réalisé pendant le stage, les fonctionnalités implémentées, les défis techniques surmontés, ainsi que les perspectives de développement futur de la plateforme LearnExpert.

\section{Réalisations et État Actuel du Projet}

\subsection{Fonctionnalités Implémentées}
Au terme de ce stage de quatre semaines, les fonctionnalités suivantes ont été implémentées avec succès :

\begin{itemize}
  \item \textbf{Architecture de base} : Mise en place d'une architecture microservices robuste et évolutive
  \item \textbf{Site vitrine} : Développement complet du site vitrine avec landing page interactive
  \item \textbf{Système d'authentification} : Implémentation de l'authentification des utilisateurs via Supabase
  \item \textbf{Dashboard utilisateur} : Création d'une interface de tableau de bord personnalisée
  \item \textbf{Éditeur de code interactif} : Intégration d'un éditeur Monaco avec coloration syntaxique et exécution en temps réel
  \item \textbf{Système de navigation} : Développement d'une navigation contextuelle et intuitive
  \item \textbf{Base de données structurée} : Migration des données scrappées et nettoyées vers une structure JSON optimisée
  \item \textbf{Pipeline de traitement de données} : Mise en place d'un système de nettoyage utilisant des modèles LLM
\end{itemize}

Ces réalisations constituent les fondations solides sur lesquelles la plateforme pourra continuer à se développer.

\subsection{Défis Techniques Surmontés}
Plusieurs défis techniques majeurs ont été rencontrés et surmontés au cours du développement :

\begin{itemize}
  \item \textbf{Séparation du code et du texte} : La difficulté initiale à distinguer automatiquement le code explicatif du code exécutable dans les données scrappées a été résolue grâce à l'utilisation innovante de modèles de langage large (LLM) en parallèle.
  
  \item \textbf{Performances de l'éditeur de code} : L'intégration de l'éditeur Monaco dans l'application Next.js a posé des défis de performance qui ont été résolus par une stratégie de chargement dynamique et de mise en cache.
  
  \item \textbf{Cohérence des données} : La gestion de la cohérence des données à travers les différents microservices a nécessité la mise en place d'un système robuste de validation et de synchronisation.
  
  \item \textbf{Optimisation du rendu côté client} : Les animations et transitions fluides ont été implémentées tout en maintenant d'excellentes performances, grâce à des techniques d'optimisation comme le code splitting et la lazy loading.
\end{itemize}

Ces défis ont non seulement été surmontés, mais ils ont également permis d'enrichir l'architecture et les fonctionnalités de la plateforme.

\section{Développements Futurs}

\subsection{Fonctionnalités Planifiées}
Plusieurs fonctionnalités ont été identifiées pour les phases futures du développement :

\begin{itemize}
  \item \textbf{Système de recommandation IA} : Implémentation d'un système de recommandation personnalisé basé sur l'IA pour suggérer des cours adaptés au niveau et aux intérêts de chaque utilisateur.
  
  \item \textbf{Mode collaboratif} : Ajout de fonctionnalités permettant à plusieurs utilisateurs de travailler simultanément sur le même projet ou exercice de code.
  
  \item \textbf{Gamification avancée} : Intégration d'éléments de gamification (badges, classements, défis) pour augmenter l'engagement des utilisateurs.
  
  \item \textbf{Module d'évaluation des compétences} : Développement d'un système d'évaluation automatisé des compétences techniques acquises.
  
  \item \textbf{Support multilingue} : Extension de la plateforme pour supporter plusieurs langues (anglais, espagnol, etc.).
  
  \item \textbf{Applications mobiles natives} : Développement d'applications mobiles natives pour iOS et Android.
\end{itemize}

\subsection{Améliorations Techniques Envisagées}
Sur le plan technique, plusieurs améliorations sont prévues :

\begin{itemize}
  \item \textbf{Optimisation des performances} : Amélioration continue des temps de chargement et de la réactivité de l'interface.
  
  \item \textbf{Renforcement de la sécurité} : Mise en place de tests de pénétration réguliers et amélioration des mécanismes de sécurité.
  
  \item \textbf{Infrastructure serverless} : Migration progressive vers une architecture serverless pour certains microservices.
  
  \item \textbf{Tests automatisés} : Augmentation de la couverture des tests automatisés pour garantir la stabilité de la plateforme.
  
  \item \textbf{Monitoring avancé} : Implémentation d'outils de monitoring et d'analyse en temps réel pour surveiller les performances et l'expérience utilisateur.
\end{itemize}

\subsection{Roadmap d'Évolution}
Une roadmap d'évolution a été établie pour les 12 prochains mois :

\begin{itemize}
  \item \textbf{Phase 1 (3 mois)} : Finalisation des fonctionnalités de base, lancement en production et première phase de test utilisateur.
  
  \item \textbf{Phase 2 (3-6 mois)} : Implémentation du système de recommandation IA, amélioration de l'éditeur de code et développement du mode collaboratif.
  
  \item \textbf{Phase 3 (6-9 mois)} : Intégration de la gamification avancée, du module d'évaluation des compétences et support multilingue initial.
  
  \item \textbf{Phase 4 (9-12 mois)} : Développement des applications mobiles natives et extension des contenus éducatifs.
\end{itemize}

Cette roadmap est conçue pour être flexible et adaptable en fonction des retours utilisateurs et des priorités business qui pourraient évoluer.

\section{Bilan Personnel du Stage}

\subsection{Compétences Techniques Acquises}
Ce stage a été l'occasion d'acquérir et de renforcer de nombreuses compétences techniques :

\begin{itemize}
  \item Maîtrise approfondie de Next.js et React avec TypeScript
  \item Conception et implémentation d'architectures microservices
  \item Utilisation avancée de MongoDB et Supabase
  \item Développement de pipelines de traitement de données avec LLM
  \item Optimisation des performances d'applications web complexes
  \item Déploiement continu avec Vercel et GitHub Actions
\end{itemize}

\subsection{Compétences Transversales Développées}
Au-delà des aspects purement techniques, ce stage a également permis de développer des compétences transversales essentielles :

\begin{itemize}
  \item Gestion de projet agile et planification efficace
  \item Communication technique et travail en équipe
  \item Résolution de problèmes complexes
  \item Autonomie et prise d'initiative
  \item Veille technologique et apprentissage continu
\end{itemize}

\subsection{Apprentissages Professionnels}
Cette expérience en entreprise a offert des apprentissages précieux sur le plan professionnel :

\begin{itemize}
  \item Compréhension des cycles de développement logiciel en environnement professionnel
  \item Sensibilisation aux enjeux business et aux attentes des utilisateurs
  \item Adaptation aux contraintes de temps et de ressources
  \item Importance de la documentation et du partage de connaissances
  \item Équilibre entre perfectionnisme technique et pragmatisme
\end{itemize}

Ce stage chez IAAI a constitué une étape déterminante dans mon parcours professionnel, m'offrant une vision concrète des défis et opportunités du développement de plateformes éducatives innovantes. 

\chapter{Chapitre 7 : Conclusion et Perspectives}
\thispagestyle{fancy}
\newpage

Ce stage de quatre semaines chez LearnTech Solutions m'a permis de participer activement au développement d'une plateforme e-learning innovante. Au cours de cette période, j'ai pu mettre en pratique mes connaissances théoriques et acquérir de nouvelles compétences techniques dans un environnement professionnel stimulant.

\section{Synthèse des Réalisations}

Durant ces quatre semaines, les principales réalisations ont été :
\begin{itemize}
  \item La conception de l'architecture microservices pour la plateforme
  \item La participation au développement des interfaces utilisateur
  \item L'implémentation de systèmes de traitement et nettoyage de données
  \item La contribution à l'optimisation des performances et à l'amélioration de l'expérience utilisateur
\end{itemize}

\section{Compétences Acquises}

Ce stage m'a permis de développer de nombreuses compétences :
\begin{itemize}
  \item \textbf{Compétences techniques :} Approfondissement des connaissances en développement web moderne (Next.js, API REST), en architecture microservices et en traitement de données avec des modèles LLM
  \item \textbf{Compétences méthodologiques :} Maîtrise des outils de gestion de projet agile, de versionnement et de déploiement continu
  \item \textbf{Compétences transversales :} Amélioration des capacités de communication, de travail en équipe et d'adaptation à un environnement professionnel
\end{itemize}

\section{Perspectives}

Ce projet ouvre plusieurs perspectives intéressantes :
\begin{itemize}
  \item \textbf{Pour l'entreprise :} Finalisation et commercialisation de la plateforme e-learning, avec des fonctionnalités supplémentaires comme l'intelligence artificielle pour la personnalisation des parcours d'apprentissage
  \item \textbf{Pour ma carrière :} Approfondissement des compétences en développement full-stack et en architecture de systèmes complexes
\end{itemize}

En conclusion, ce stage a été une expérience extrêmement enrichissante qui m'a permis d'approfondir mes connaissances techniques tout en développant ma compréhension des enjeux liés au développement de solutions éducatives numériques. Les compétences acquises et l'expérience professionnelle seront des atouts précieux pour ma future carrière.

\appendix
\chapter{Annexes}
\thispagestyle{fancy}
[Diagrammes UML, exemples de code, captures d'écran de l'interface]

\end{document}
