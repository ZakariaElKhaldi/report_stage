\chapter{Chapitre 1 : Présentation de l'Entreprise et Cadre du Stage}
\thispagestyle{fancy}
\newpage

Ce chapitre présente l'entreprise IAAI, le cadre général du stage, ainsi que les objectifs fixés pour cette période d'apprentissage professionnel.

\section{Présentation de IAAI}

\subsection{Historique et Domaine d'Activité}
IAAI (Intelligence Artificielle et Apprentissage Interactif) est une entreprise innovante spécialisée dans le développement de solutions éducatives numériques basées sur l'intelligence artificielle. Fondée en 2020, IAAI s'est rapidement positionnée comme un acteur émergent dans le domaine de l'e-learning et des technologies éducatives.

L'entreprise se concentre sur la création de plateformes d'apprentissage en ligne qui utilisent l'IA pour personnaliser l'expérience d'apprentissage et optimiser l'acquisition de compétences techniques, particulièrement dans le domaine de la programmation et des technologies web.

\subsection{Vision et Mission}
La vision d'IAAI est de démocratiser l'accès à une éducation technique de qualité grâce à des plateformes innovantes et accessibles. 

Sa mission est double :
\begin{itemize}
  \item Développer des plateformes d'apprentissage qui s'adaptent automatiquement au rythme et au style d'apprentissage de chaque utilisateur
  \item Rendre l'apprentissage de la programmation et des technologies plus engageant et efficace grâce à l'utilisation de l'IA et des technologies interactives
\end{itemize}

\subsection{Organisation Structurelle}
IAAI est organisée en plusieurs départements complémentaires :
\begin{itemize}
  \item \textbf{Département R\&D} : Chargé de la recherche en IA et en sciences de l'apprentissage
  \item \textbf{Département Technique} : Responsable du développement des plateformes et des applications
  \item \textbf{Département Pédagogique} : En charge de la création et de la curation des contenus éducatifs
  \item \textbf{Département Marketing et Commercial} : Gestion de la stratégie de croissance et des relations clients
\end{itemize}

L'entreprise compte une vingtaine de collaborateurs, principalement des développeurs, des experts en IA, des spécialistes en contenu pédagogique et des designers UX/UI.

\section{Cadre Général du Stage}

\subsection{Objectifs du Stage}
Ce stage s'inscrit dans le cadre du développement de la nouvelle plateforme e-learning LearnExpert d'IAAI. Les objectifs principaux étaient :

\begin{itemize}
  \item Participer à la conception de l'architecture globale de la plateforme
  \item Développer les interfaces utilisateur principales et le système de navigation
  \item Mettre en place un système d'acquisition et de traitement de données pour les contenus éducatifs
  \item Intégrer des fonctionnalités interactives pour l'apprentissage de la programmation
  \item Contribuer au déploiement et à l'optimisation de la plateforme
\end{itemize}

Ces objectifs s'inscrivent dans une vision plus large de création d'une plateforme complète, modulaire et évolutive, destinée à servir de base pour les futurs produits de l'entreprise.

\subsection{Environnement de Travail}
Le stage s'est déroulé dans les locaux d'IAAI, au sein d'une équipe de développement composée de :

\begin{itemize}
  \item Un chef de projet technique
  \item Deux développeurs full-stack seniors
  \item Un designer UX/UI
  \item Un spécialiste en contenu éducatif
\end{itemize}

L'environnement de travail était basé sur des méthodologies agiles, avec des sprints hebdomadaires, des réunions quotidiennes de synchronisation, et un système de revue de code. Les outils principaux utilisés incluaient :

\begin{itemize}
  \item GitHub pour la gestion de versions et la collaboration
  \item Jira pour la gestion des tâches et le suivi des bugs
  \item Figma pour les maquettes et le design
  \item Slack et Discord pour la communication d'équipe
\end{itemize}

\subsection{Planning et Méthodologie}
Le stage s'est déroulé sur une période de quatre semaines, organisées selon un planning structuré :

\begin{itemize}
  \item \textbf{Semaine 1} : Conception et architecture de la plateforme
  \item \textbf{Semaine 2} : Développement du site vitrine et de la landing page
  \item \textbf{Semaine 3} : Développement des interfaces utilisateur et traitement des données
  \item \textbf{Semaine 4} : Finalisation, déploiement et optimisations
\end{itemize}

La méthodologie adoptée était une version adaptée de Scrum, avec des cycles de développement courts permettant des ajustements rapides et une flexibilité face aux changements de priorités. Chaque sprint débutait par une session de planification et se terminait par une démonstration des fonctionnalités implémentées, suivie d'une rétrospective pour identifier les points d'amélioration. 