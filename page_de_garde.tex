\documentclass[a4paper,12pt]{report}

% Encodage et fontes (XeLaTeX)
\usepackage{fontspec}
  \setmainfont{TeX Gyre Termes}           % Police principale (sérif)
  \setsansfont{TeX Gyre Adventor}         % Police secondaire (sans)
\renewcommand{\familydefault}{\sfdefault} % Définit la police par défaut au sans

\usepackage{polyglossia}
  \setmainlanguage{french}

\usepackage{xcolor,graphicx}
\usepackage[top=2cm,bottom=2cm,left=1.3cm,right=1.3cm]{geometry}
\usepackage{tikz}
  \usetikzlibrary{calc,decorations.pathmorphing}
\usepackage[many]{tcolorbox}

% Couleurs
\definecolor{gold}{RGB}{184,134,11}
\definecolor{navy}{RGB}{0,33,71}
\definecolor{lightnavy}{RGB}{245,248,255}
\definecolor{sepgray}{RGB}{160,160,160}


% Taille du titre
\newcommand{\HUGEtitle}{\fontsize{36}{44}\selectfont}

% Définition de la commande shadowtext manquante
\newcommand{\shadowtext}[1]{%
  \begingroup
  \setbox0=\hbox{#1}%
  \leavevmode\rlap{\raisebox{1.5pt}{\makebox[\wd0]{\textcolor{black!40}{#1}}}}%
  #1%
  \endgroup
}

% Avant \begin{document}, ajoutez :
\usepackage{tikz}
\usepackage{transparent}
\newcommand{\fadetext}[3][]{%
  % #1 : options TikZ, #2 : fade color, #3 : texte
  \begin{tikzpicture}[baseline=(T.base),#1]
    \node[inner sep=0pt] (T) {\bfseries #3};
    \begin{scope}[on behind layer]
      \node[opacity=0.2,scale=1.2,text=black] at (T) {#3};
    \end{scope}
    \fill[top color=#2!80,bottom color=white!20,opacity=0.5]
      ($(T.south west)-(0.1em,0.1em)$) rectangle ($(T.north east)+(0.1em,0.1em)$);
  \end{tikzpicture}%
}

\begin{document}
\thispagestyle{empty}

\begin{titlepage}
  % Bandeaux supérieurs « premium » avec dégradé et courbes
  \begin{tikzpicture}[remember picture,overlay]
    % Dégradé de bleu marine du haut vers le centre
    \shade[top color=navy!90!black, bottom color=navy!60!black]
      (current page.north west) rectangle ++(\paperwidth,-2cm);
    % Bande dorée légèrement arrondie en dessous
    \fill[gold]
      ($(current page.north west)+(0,-2cm)$) 
      rectangle ++(\paperwidth,-0.5cm);
    % Courbe douce pour marquer la transition
    \draw[white, line width=1.5pt]
      plot[smooth, tension=0.7]
        coordinates {
          ($(current page.north west)+(0,-2cm)$)
          ($(current page.north west)+(0.2\paperwidth,-2.2cm)$)
          ($(current page.north east)+(-0.2\paperwidth,-2.2cm)$)
          ($(current page.north east)+(0,-2cm)$)
        };
  \end{tikzpicture}


  % Logos (ajustés et abaissés)
  \vspace*{0.7cm}  % espace avant logos
  \begin{minipage}[t]{0.48\textwidth}
    \includegraphics[height=2cm]{ministere.jpg}
  \end{minipage}\hfill
  \begin{minipage}[t]{0.48\textwidth}
    \flushright
    \includegraphics[height=2cm]{bts_logo.png}
  \end{minipage}

  % Titre principal
  \vspace*{2cm}
  \begin{center}
    % Ligne 1 : Gros titre avec ombre légère
    {\HUGEtitle\bfseries\textcolor{navy}{\shadowtext{STAGE de FIN d'ÉTUDES}}}\\[16pt]
    % Ligne 2 : BTS en petites capitales, encadré
    \tikz[baseline]\node[draw=gold,thick,rounded corners=2pt,inner xsep=6pt,inner ysep=2pt]
      {\Large\scshape Brevet de Technicien Supérieur (BTS)};\\[12pt]
    % Ligne 3 : Filière en italique clair
    {\normalsize\itshape Filière~: \textcolor{sepgray}{Développement des Systèmes d'Information}}
  \end{center}


  % Séparateur décoratif
  \vspace*{1.5cm}
  \begin{center}
    \begin{tikzpicture}
      % Trait principal gris
      \draw[line width=1pt, color=sepgray] (0,0) -- (8,0);
      % Trait secondaire doré, légèrement en dessous
      \draw[line width=2pt, color=gold] (0,-2pt) -- (8,-2pt);
    \end{tikzpicture}
  \end{center}


  % Sous-titre encadré « premium » et texte centré
  \vspace*{1cm}
  \begin{center}
    \begin{tcolorbox}[
      enhanced,
      frame hidden,                             % pas de contour superflu
      borderline west={4pt}{0pt}{gold},         % bord gauche épais doré
      interior style={                          % dégradé vertical subtil
        top color=lightnavy!70!white,
        bottom color=lightnavy!30!white
      },
      boxrule=0pt,                              % contour principal désactivé
      arc=8pt,                                  % coins arrondis
      drop fuzzy shadow={black!20},             % ombre portée simplifiée
      width=0.8\textwidth,                      % largeur du cadre
      enlarge left by=3mm,                      % légers débordements pour l'élégance
      enlarge right by=3mm,
      leftrule=1pt,                             % fine ligne or à gauche
      colback=lightnavy!25!white,               % fond clair nuancé
      fontupper=\bfseries\Large\centering,      % texte en gras, plus grand et centré
      before upper=\vspace*{6pt},               % marges internes haut
      after upper=\vspace*{6pt}                 % marges internes bas
    ]
      Conception et Développement\\
      d'une Plateforme E-Learning
    \end{tcolorbox}
  \end{center}


  % Auteur et encadrant (sans erreurs d'affichage)
  \vspace*{2.5cm}
  \begin{center}
    \begin{minipage}[t]{0.45\textwidth}
      \raggedright
      {\large\bfseries Réalisé par}\\[4pt]
      {\Large\scshape Zakaria El Khaldi}\\[6pt]
      {\color{sepgray}\rule{0.6\linewidth}{0.5pt}}
    \end{minipage}\hfill
    \begin{minipage}[t]{0.45\textwidth}
      \raggedleft
      {\large\bfseries Encadré par}\\[4pt]
      {\Large\scshape Pr. Ayman Naim}\\[6pt]
      {\color{sepgray}\rule{0.6\linewidth}{0.5pt}}
    \end{minipage}
  \end{center}


  % Pied de page amélioré « premium »
  \vfill
  \begin{center}
  \vspace*{1.7cm}
    % Double ligne décorative
    {\color{sepgray}\rule{0.3\linewidth}{1pt}}\\[-1pt]
    {\color{gold}\rule{0.15\linewidth}{2pt}}\\[8pt]
    
    % Texte centré et hiérarchisé
    {\large\bfseries Année académique : 2024/2025}\\[4pt]
    {\small\itshape Centre BTS Al-Kendi \textbullet{} Casablanca, Maroc}
  \end{center}


  % Filet bas « premium » avec vague décorative
  \begin{tikzpicture}[remember picture,overlay]
    % Bande principale dorée
    \fill[gold!90!black]
      (current page.south west) rectangle ++(\paperwidth,0.4cm);
    % Bande secondaire plus fine bleu marine
    \fill[navy]
      (current page.south west) ++(0,0.4cm) rectangle ++(\paperwidth,0.1cm);
    % Vague légère au-dessus du filet
    \draw[
      decorate,
      decoration={snake, amplitude=1pt, segment length=8pt},
      line width=0.6pt,
      color=navy!70
    ]
      (current page.south west) ++(0,0.5cm) -- ++(\paperwidth,0);
  \end{tikzpicture}

\end{titlepage}

\end{document} 